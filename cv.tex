---
title: "Curriculum Vitae"
author: "Daniel Paul O'Donnell"
date: \today
geometry: margin=1in
fontsize: 12pt
# mainfont: "Linux Libertine O" # or any font you want
link-citations: true
colorlinks: true
urlcolor: blue
# documentclass: article
header-includes:
  - \usepackage[hyphens]{url}
  - \usepackage{hyperref}
  - \hypersetup{breaklinks=true}
  - \urlstyle{same}
  - \usepackage{needspace}
  - \newcommand{\sectionbreak}{\needspace{5\baselineskip}}

---

\begin{document}

\begin{center}
\LARGE \textbf{Daniel Paul O'Donnell} \\
\vspace{0.25em}
\normalsize Curriculum Vitae
\end{center}

\sectionbreak{}
---

# Professional Metadata
- **Email**: daniel.odonnell@uleth.ca
- **Phone**: +1 (403) 393-2539
- **ORCID**: [0000-0002-0127-4893](https:/\allowbreak{}/\allowbreak{}orcid.org/\allowbreak{}0000-0002-0127-4893)
- **Zenodo Repository**: https:/\allowbreak{}/\allowbreak{}zenodo.org/\allowbreak{}communities/\allowbreak{}dpodrepository
- **Institutional Repository**: https:/\allowbreak{}/\allowbreak{}leth.ca/\allowbreak{}dspace/\allowbreak{}handle/\allowbreak{}10133/\allowbreak{}3557

\sectionbreak{}
---
# Education

**Summary**: O'Donnell completed his undergraduate studies at the University of Toronto in English and Medieval Latin, followed by an MA and PhD at Yale University in English. His doctoral research focused on variation in Old English poetry, supervised by Fred C. Robinson.

- **PhD, English**, Yale University, 1996
  *Dissertation*: “Manuscript Variation in Multiple Recension Old English Poetic Texts: The Technical Problem and Poetical Art."  https:/\allowbreak{}/\allowbreak{}doi.org/\allowbreak{}10.5281/\allowbreak{}zenodo.1171976
  *Supervisor*: Fred C. Robinson
- **MA, English**, Yale University, 1991
- **BA with distinction**, St. Michael’s College, University of Toronto, 1989
  *Specialisation*: English Language and Literature
  *Minors*: Medieval Latin (official), Celtic Studies (completed requirements but not formally recorded)

\sectionbreak{}
# Languages

**Summary**: O'Donnell works in a range of modern and historical languages relevant to medieval studies and digital scholarship. These include English (native), Dutch (spoken and read), French (reading and conversational), and German (reading), as well as a variety of medieval and classical languages, such as Old English, Old Norse, Latin, Old Frisian, and Gothic.

- English (native)
- Dutch (read, spoken, and aural)
- French (reading and aural strong; improving spoken accuracy in preparation for CERF B2 certification)
- German (reading)

\sectionbreak{}
# Academic Positions

**Summary**: Since 1997, O'Donnell has held a continuing faculty position at the University of Lethbridge, where he currently holds the rank of Professor. He has also taught at Louisiana State University, the University of York, and Yale University, and has held adjunct and affiliate roles at the University of Saskatchewan and within the University Library.

- **Professor of English (Tenured)**, University of Lethbridge, 2010–present
- **Affiliate Member**, Prentice Institute for Global Population, 2020–
- **Adjunct Professor**, University of Saskatchewan (Graduate College), 2019–2024
- **Associate Member**, University Library Academic Staff, University of Lethbridge, 2015–
- **Professor of English (Offered and Declined)**, University of Saskatchewan, 2018
- **Associate Professor of English**, University of Lethbridge, 2002–2010
- **Assistant Professor of English**, University of Lethbridge, 1997–2002
- **Tutor**, University of York /\allowbreak{} Workers’ Educational Association (UK), 1997
- **Visiting Assistant Professor**, Louisiana State University, 1994–1995
- **Teaching Fellow**, Yale University, 1991–1992
- **Research Assistant**, Dictionary of Old English, University of Toronto, 1987–1989

\sectionbreak{}
# Academic Leadership

**Summary**: O'Donnell has led multiple academic and professional organisations, including CAFA, ULFA, Force11, GO::DH, and the TEI. He has chaired major digital infrastructure and policy boards, founded scholarly and advocacy groups, and served as department chair at Lethbridge. In these roles, he has consistently taken leadership during moments of structural transition and policy change.

- **Past President**, Confederation of Alberta Faculty Associations (CAFA), 2024–2025
  Continued strategic advocacy; transitioned president role following reorganisation.
- **President**, CAFA, 2023–2024
  Led reorganisation after loss of major member (AASUA); founded Western Regional Council; initiated successful provincial lobbying on Bill 18.
- **Past President**, University of Lethbridge Faculty Association (ULFA), 2023–2024
  Supported transition to new leadership; advised on grievance resolution and governance rebuilding.
- **President**, ULFA, 2021–2023
  Led six-week job action (92% strike vote, 91% ratification); rebuilt faculty governance post-strike; expanded communication strategy.
- **Vice-President**, ULFA, 2020–2021
  Developed internal governance models; supported bargaining team formation.
- **Chief Spokesperson and Bargaining Chair**, ULFA, 2018–2021
  Negotiated multiple collective agreements (including Navitas MOU); launched and secured approval for job action fund.
- **President**, Force11, 2018–2019
  Oversaw launch of FORCE11 Scholarly Communication Institute (FSCI); expanded global FAIR and Open Science policy advocacy.
- **Vice-President**, Force11, 2013–2017
  Built global partnerships in scholarly communication and infrastructure.
- **Founding Director**, Lethbridge Centre for the Study of Scholarly Communication, 2015–present
  Created institutional centre supporting interdisciplinary research and publication innovation.
- **Founding Chair**, Global Outlook::Digital Humanities (GO::DH), 2012–2015
  Founded equity-centred digital humanities network; initiated multilingual and regional DH programs.
- **Co-President**, Canadian Society for Digital Humanities (CSDH/\allowbreak{}SCHN), 2010–2013
  Coordinated national DH strategy and membership growth.
- **Chair**, Digital Initiatives Advisory Board, Medieval Academy of America, 2009–2012
  Guided MAA infrastructure strategy; advised on journal and archive digitisation.
- **Chair and CEO**, Text Encoding Initiative (TEI), 2006–2011
  Restructured TEI governance; launched member-led consortium; resolved internal staff conflict; published revised TEI Guidelines.
- **Founding Director**, Digital Medievalist Project, 2003–2009
  Established peer-reviewed journal and infrastructure for medievalists working in digital media.
- **Chair**, Department of English, University of Lethbridge, 2005–2008, 2023–present
  Led department through curriculum overhaul, review, and hiring plan during institutional crisis; implemented new governance model.

\sectionbreak{}
# Current Major Projects

**Summary**: O'Donnell is Principal Investigator on several research initiatives focused on scholarly communication, data infrastructure, and academic governance. These projects are supported by SSHRC and international partners including CERN, DARIAH, and CARL. His current work addresses how humanities research resists or reimagines dominant models of data, infrastructure, and institutional accountability.

- **Resistance to Data: Humanities Approaches to Research Classification and Infrastructure**
  *(2022–present; SSHRC Insight Grant 2025–2029, PI, $300,000; SSHRC Partnership Development Grant 2022–2025, PI, $200,000)*
  Umbrella initiative investigating epistemic and institutional resistance to structured data in the humanities. Incorporates and extends the earlier *Good Things Come in Small Packages* project. Includes strands in multilingual infrastructure, human-centred metadata, and knowledge equity. Collaborative partners include CERN, CARL, and DARIAH-EU.
- **#safespaces? Academic Freedom and Inclusivity in an Age of Polarisation**
  *(2024–; SSHRC Insight Development Grant – applied for, PI)*
  Investigates how political polarisation, identity politics, and campus speech discourses intersect with evolving interpretations of academic freedom in Canada and abroad.
- **Humanities Innovation Lab**
  *(Internal initiative; PI)*
  Conceptual pilot adapting laboratory-based research models to the humanities. Emphasises collaborative supervision, research planning, and the integration of graduate students into structured team-based environments.
- **Lethbridge Journal Incubator**
  *(2012–2024; SSHRC Aid to Scholarly Journals Grant, PI; internal funding)*
  A training and mentorship initiative in scholarly communication. Offers graduate students editorial experience and exposure to open infrastructure and publishing innovation.
- **The Visionary Cross in Anglo-Saxon England**
  *(2005–present; SSHRC Standard Research Grant 2005–2008, PI; SSHRC Insight Grant 2014–2017, PI, $282,190; Mitacs support; Chinook Summer Research Grant, Government of Alberta, $6,000, 2025)*
  A long-running digital edition and visualisation project focused on the Ruthwell Cross and related Insular artefacts. Integrates philological, archaeological, and imaging methodologies to interpret monumental Old English inscriptions. Outputs include multi-format visualisation environments, digital philology, and graduate student training.

\sectionbreak{}
# Funding and Prizes

**Summary**: O'Donnell has received approximately CA$4 million in research funding as Principal Investigator and Co-applicant from external agencies since 2000 (CA$2 million as PI). His work spans scholarly communication, early Medieval English studies, research infrastructure, and innovation in academic publishing.

\sectionbreak{}
## External Research Grants and Prizes

- SSHRC Impact Prize (Connections) (780-2025-00021), Nominee. $50,000 (under review)
- SSHRC Insight Development Grant, *#safespaces? Academic Freedom and Inclusivity in an Age of Polarisation*, 2025-2027, PI, $78,000 (under review)
- SSHRC Insight Grant (435-2025-0705), *Resistance to Data: Understanding Data Use, Data Management, and Data Infrastructure in the "Traditional" Humanities through Historical, Comparative, and Ethnographic Study*, 2025–2029, PI, $300,000
- SSHRC Insight Grant, *Canterbury Tales Project*, 2021–2026, Co-applicant, $340,000
- SSHRC Partnership Development Grant (890-2020-0095), *Good Things Come in Small Packages: A Grassroots Community of Practice for Open and FAIR Humanities Data Practices*, 2021–2023, PI, $200,000
- SSHRC Aid to Scholarly Journals (651-2021-0150), *Digital Studies /\allowbreak{} Le champ numérique*, 2022–2025, PI, $30,000
- SSHRC Aid to Scholarly Journals (651-2018-0062), *Digital Studies /\allowbreak{} Le champ numérique*, 2019–2022, PI, $90,000
- SSHRC Connections Grant (611-2019-0499), *Canada-LATAM Workshop on Open and Inclusive Access to research*, 2020–2021, PI, $25,000
- Sloan Foundation (G-2020-13999), *Developing Open Science during the COVID emergency: A community-based investigation and support network: Reimagining Education Practices for Open (REPO)*, 2020–2021, PI, US$50,000
- SSHRC Partnership Development Grant (890-2016-0081), *Future Commons: Transforming Scholarly Communication through Collective Action*, 2017–2019, PI, $200,000
- CFI John R. Evans Leaders Fund (32819), *What Goes Around: The Visionary Cross Digital Library*, 2017-2019, PI, $86,937
- SSHRC Insight Grant (435-2015-1119), *What Goes Around: Editing the Anglo-Saxon Visionary Cross Cultural Matrix*, 2015–2019, PI, $233,224
- Mellon Foundation, *Reading Peer Review*, 2017–2018, Co-PI, US$99,000
- SSHRC Aid to Scholarly Journals (651-2014-0138), *Digital Studies /\allowbreak{} Le champ numérique*, 2014–2017, PI, $46,225
- SSHRC Insight Grant, *Canterbury Tales Project Phase 2*, 2013–2017, Co-applicant, $471,000
- Helmsley Charitable Trust (via Force11), 2015, drafting applicant, US$424,000
- Mitacs GlobalLink (Visionary Cross), 2015, In-kind
- Gordon and Betty Moore Foundation (Force11), 2014–2015, Co-applicant, US$150,000
- GRAND Startup Grant (G-CI-14-LB-01), *DigiCultH: Engaging with Digital Cultural Heritage Objects*, 2014–2015, Co-PI, $9,640
- SSHRC Standard Research Grant (410-2010-1474), *Crossroads: editing the Visionary Cross Matrix in Anglo-Saxon England*, 2010–2014, PI, $62,430
- Mellon Foundation, Scholarly Communications Program (GP1.2008), *TEI Tite: Creating a Benefit of Membership to Support Standards Development*, 2008, PI, US$30,723
- SSHRC ITST Grant (849-2003-0003), *The Digital Medievalist Project: A Community of Practice Network for Image, Text, Sound and Technology Research*, 2005–2006, PI, $27,490

\sectionbreak{}
## Internal and Institutional Grants (selected)

- UCLA (Force11 Scholarly Communications Institute), 2020–2022, PI, US$190,000
- CREDO Grant, *Visionary Cross*, 2008–2010, $23,000
- CREDO Grant, *Humanities Data*, 2019–2022, $25,000
- Lethbridge Journal Incubator RAships (multiple years), in-kind, $21,900–$29,200 per round
- University of Lethbridge Internal SSHRC and Research Fund awards, various years
- Teaching Development Fund, *The Unessay*, 2013–2014, $2,700
- Internal SSHRC Grant, *Global Outlook::DH*, 2012, $5,000

\sectionbreak{}
## Awards, Fellowships, and Honours

- University of Lethbridge Teaching Award Nominee 2015–2016, 2016-2017.
- Honourable Mention, MLA Prize for a Scholarly Edition (*Cædmon’s Hymn*), 2007
- Mellon Fellowship in the Humanities (Dissertation Fellowship), 1993–1994, US$11,000
- SSHRC Doctoral Fellowship, 1992–1994, $14,000 per year
- Yale University Fellowship, 1992–1993, US$16,000 + tuition
- University of California, Los Angeles, University Fellowship, 1989–1993, US$36,000 + tuition (declined)
- Mellon Fellowship in the Humanities, 1989–1991, US$22,000 + tuition
- Yale University Fellowship, 1989–1991, US$16,000 + tuition (declined)
- C.L. Burton In-course Scholarship, St. Michael’s College, University of Toronto, 1988–1989, $1,500 (declined)
- In-course Scholarship, St. Michael’s College, University of Toronto, 1987–1988, $1,500 (declined)

\sectionbreak{}
# Publications
**Summary**: O'Donnell's publications include scholarly editions, monographs, peer-reviewed articles, invited chapters, reviews, working papers, and public-facing essays. His work spans early medieval literature, scholarly communication, and digital humanities. In addition to conventional print outputs, he has co-developed a number of alternate and novel forms of dissemination, including digital editions, interactive 3D cultural heritage projects, and experimental publishing platforms such as the Visionary Cross Project and the Lethbridge Journal Incubator.

**Note**: In publications, the following symbols are used: (r) = refereed article; (c) = corresponding author; (s) = student co-authors; (i) = invited submission. An asterisk after an author's name indicates the person was a student at the time of publication.

\sectionbreak{}
## Books and Editions

- Eve, Martin Paul, Cameron Neylon, Daniel Paul O’Donnell, Samuel Moore\*, Robert Gadie\*, Victoria Odeniyi\*, and Shahina Parvin\*. 2020. *Reading Peer Review: PLOS ONE and Institutional Change in Academia*. Cambridge: Cambridge University Press. https:/\allowbreak{}/\allowbreak{}doi.org/\allowbreak{}10.1017/\allowbreak{}9781108783521 (rs)
  ISBN: 9781108486637 (hardback); 9781108783521 (ebook)
  - **Reviews**:
    - *Public Understanding of Science* 31 (7): 892–894
- O’Donnell, Daniel Paul. 2005. *Cædmon’s Hymn: A Multimedia Study, Edition and Archive*. SEENET A.8. Cambridge: D.S. Brewer /\allowbreak{} Medieval Academy of America. xxii + 261 pp. + CD-ROM.
  Internet reprint (2018): https:/\allowbreak{}/\allowbreak{}caedmon.seenet.org/\allowbreak{} (r)
  Codebase DOI: https:/\allowbreak{}/\allowbreak{}doi.org/\allowbreak{}10.5281/\allowbreak{}zenodo.1198856
  - **Honourable Mention**: 2007 MLA Prize for a Distinguished Scholarly Edition (2005–2006)
  - **Reviews**:
    - *e-data&research* 1 (2006): 1; *Medium Aevum* 75 (2006): 356–357; *Old English Newsletter* 40.1 (2006); *Speculum* 82 (2007): 223–224; *Journal of Ecclesiastical History* 58 (2007): 120–121; *Early Medieval Europe* 15 (2007): 466–469; *Textual Cultures* 2 (2007): 139–142; *JEGP* 107.2 (2008): 248–251; *Digital Medievalist* 5 (2009); *Leeds Medieval Studies* 3 (2023): 139–141

\sectionbreak{}
## Peer-Reviewed and Invited Articles and Chapters (Traditional)

- Pafumi, Davide\*, Frank Onuh\*, AKM Iftekhar Khalid\*, Morgan Pearce\*, Barbara Bordalejo, and Daniel Paul O'Donnell. 2025. “Scarlet Cloak and the Forest Adventure”: A Preliminary Study of the Impact of AI on Commonly Used Writing Tools. *International Journal of Educational Technology in Higher Education*. https:/\allowbreak{}/\allowbreak{}doi.org/\allowbreak{}10.1186/\allowbreak{}s41239-025-00505-5. (r)
- O’Donnell, Daniel Paul. 2021. “‘I Heard He Sang a Good Song’: Caedmon’s Inspiration and Medieval Dream Theory.” In *Sogni, visioni e profezie nella letteratura germanica medievale*, edited by Roberto Rosselli Del Turco, 147–172. Bibliotheca Germanica. Studi e testi 48. Allessandria: Edizioni del l’Orso. (i)
- Bliss, Heather, Inge Genee, Marie-Odile Junker, and Daniel Paul O’Donnell. 2020. “‘Credit Where Credit Is Due’: Authorship and Attribution in Algonquian Language Digital Resources.” *IDEAH*. https:/\allowbreak{}/\allowbreak{}doi.org/\allowbreak{}10.21428/\allowbreak{}f1f23564.3d64b2ed. (r)
- O’Donnell, Daniel Paul. 2020. “Critical Mass: The Listserv and the Early Online Community as a Case Study in the Unanticipated Consequences of Innovation in Scholarly Communication.” In *Digital Technology and the Practices of Humanities Research*, edited by Jennifer Edmond, 184–206. Cambridge: Open Book Publishers. Offprint: https:/\allowbreak{}/\allowbreak{}doi.org/\allowbreak{}10.5281/\allowbreak{}ZENODO.3633429. (r)
- O’Donnell, Daniel Paul. 2019. “All Along the Watchtower: Intersectional Diversity as a Core Intellectual Value in the Digital Humanities.” In *Intersectionality in Digital Humanities*, edited by Barbara Bordalejo and Roopika Risam. Amsterdam: ARC. Offprint: http:/\allowbreak{}/\allowbreak{}doi.org/\allowbreak{}10.5281/\allowbreak{}zenodo.3580235. (r)
- Esau, Paul\*, Carey Viejou\*, Sylvia Chow\*, Kimberly Dohms\*, Steve Firth\*, Jarret McKinnon\*, Dorethea Morrison\*, Reed Parsons\*, Courtney Rieger\*, Vanja Spiric\*, Elaine Toth\*, Kayla Ueland\*, Rumi Graham, and Daniel Paul O’Donnell. 2018. “‘Let’s Start a Journal!’: The Multidisciplinary Graduate Student Journal as Educational Opportunity.” *The Journal of Electronic Publishing* 21 (1). https:/\allowbreak{}/\allowbreak{}doi.org/\allowbreak{}10.3998/\allowbreak{}3336451.0021.109. (rsc)
- O’Donnell, Daniel Paul, Carey Viejou\*, Sylvia Chow\*, Kimberly Dohms\*, Paul Esau\*, Steve Firth\*, Rumi Graham, et al. 2018. “Zombie Journals: Designing a Technological Infrastructure for a Precarious Graduate Student Journal.” *Scholarly and Research Communication* 9 (2). https:/\allowbreak{}/\allowbreak{}src-online.ca/\allowbreak{}index.php/\allowbreak{}src/\allowbreak{}article/\allowbreak{}view/\allowbreak{}296/\allowbreak{}548. http:/\allowbreak{}/\allowbreak{}dx.doi.org/\allowbreak{}10.22230/\allowbreak{}src.2018v9n2a296. (rsc)
- O’Donnell, Daniel Paul, Matteo Callieri, Marco Dellepiane, Catherine Karkov, Dot Porter, and Roberto Rosselli Del Turco. 2018. “Archaeology in the Study: Scanning Anglo-Saxon Artifacts in the Visionary Cross Project.” *Wiðowinde* 185 (Spring): 21–27. Offprint: https:/\allowbreak{}/\allowbreak{}doi.org/\allowbreak{}10.5281/\allowbreak{}zenodo.1208167. (ic)
- Tennant, Jonathan P., Jonathan M. Dugan, Daniel Jacques Damien Graziotin, François Waldner, Daniel Mietchen, Yehia Elkhatib, Lauren B. Collister, et al. 2017. “A Multi-Disciplinary Perspective on Emergent and Future Innovations in Peer Review.” *F1000Research* 6 (1151). http:/\allowbreak{}/\allowbreak{}doi.org/\allowbreak{}10.12688/\allowbreak{}f1000research.12037.2. (rs)
- Moore, Samuel\*, Cameron Neylon, Martin Paul Eve, Daniel Paul O’Donnell, and Damian Pattinson. 2017. “‘Excellence R Us’: University Research and the Fetishisation of Excellence.” *Palgrave Communications* 3 (January). Nature Publishing Group. http:/\allowbreak{}/\allowbreak{}doi.org/\allowbreak{}10.1057/\allowbreak{}palcomms.2016.105. (rs)
- O’Donnell, Daniel Paul. 2016. “The Bird in Hand: Humanities Research in the Age of Open Data.” In *The State of Open Data: A Selection of Analyses and Articles about Open Data*, edited by Figshare, 34–35. London: Digital Science. Offprint: https:/\allowbreak{}/\allowbreak{}doi.org/\allowbreak{}10.5281/\allowbreak{}zenodo.1470821. (i)
- O’Donnell, Daniel Paul, Jessica Bay\*, Emma Dering\*, Matt Gal\*, Virgil Grandfield\*, Heather Hobma\*, and Gurpreet Singh\*. 2016. “The Third Academic Freedom.” *Light on Teaching*, 4–9. Lethbridge: Teaching Centre. Offprint: http:/\allowbreak{}/\allowbreak{}doi.org/\allowbreak{}10.5281/\allowbreak{}zenodo.3596098. (rsc)
- Hobma, Heather\*, Daniel Paul O'Donnell, Catherine Karkov, Sally Foster, James Graham, Wendy Osborn, Roberto Rosselli Del Turco, Robert Broatch, Susan Broatch, Marco Callieri, and Matteo Dellepiane. 2016. “Modern Impact on the Fabric of the Ruthwell Cross.” *Old English Newsletter* 46.1. http:/\allowbreak{}/\allowbreak{}www.oenewsletter.org/\allowbreak{}OEN/\allowbreak{}issue/\allowbreak{}ruthwell.php. (rsc)
- Champieux, Robin, Bianca Kramer, Jeroen Bosman, Ian Bruno, Amy Buckland, Sarah Callaghan, Chris Chapman, Stephanie Hagstrom, MaryAnn E. Martone, and Daniel Paul O'Donnell. 2016. “Finding the Principles of the Commons: A Report of the Force11 Scholarly Communications Working Group.” *Collaborative Librarianship* 8 (2). http:/\allowbreak{}/\allowbreak{}digitalcommons.du.edu/\allowbreak{}collaborativelibrarianship/\allowbreak{}vol8/\allowbreak{}iss2/\allowbreak{}5. (r)
- Kramer, Bianca, Jeroen Bosman, Marcin Ignac, Christina Kral, Tellervo Kalleinen, Pekko Koskinen, Ian Bruno, Amy Buckland, Sarah Callaghan, Robin Champieux, Chris Chapman, Stephanie Hagstrom, MaryAnn Martone, Fiona Murphy, and Daniel Paul O'Donnell. 2016. “Defining the Scholarly Commons – Reimagining Research Communication. Report of Force11 SCWG Workshop, Madrid, Spain, February 25–27, 2016.” *Research Ideas and Outcomes* 2 (May): e9340. http:/\allowbreak{}/\allowbreak{}dx.doi.org/\allowbreak{}10.3897/\allowbreak{}rio.2.e9340. (r)
- O'Donnell, Daniel Paul, Alex Gil, Katherine Walters, and Neil Fraistat. 2015. “Only Connect: The Globalization of the Digital Humanities.” In *A New Companion to Digital Humanities*, edited by Susan Schreibman, Ray Siemens, and John Unsworth, 493–510. Wiley. https:/\allowbreak{}/\allowbreak{}doi.org/\allowbreak{}10.1002/\allowbreak{}9781118680605.ch34. (rc)
- O'Donnell, Daniel Paul, Heather Hobma\*, Sandra Cowan, Gillian Ayers\*, Jessica Bay\*, Marinus Swanepoel, Wendy Merkley, Kelaine Devine\*, Emma Dering\*, and Inge Genee. 2015. “Aligning Open Access Publication with the Research and Teaching Missions of the Public University: The Case of the Lethbridge Journal Incubator (If 'if's and 'and's Were Pots and Pans).” *Journal of Electronic Publishing* 18 (3). http:/\allowbreak{}/\allowbreak{}dx.doi.org/\allowbreak{}10.3998/\allowbreak{}3336451.0018.309. (rsc)
- Leoni, Chiara\*, Marco Callieri, Matteo Dellepiane, Daniel Paul O'Donnell, Roberto Rosselli Del Turco, and Roberto Scopigno. 2015. “The Dream and the Cross: A 3D Scanning Project to Bring 3D Content in a Digital Edition.” *Journal on Computing and Cultural Heritage*. February. https:/\allowbreak{}/\allowbreak{}doi.org/\allowbreak{}10.1145/\allowbreak{}2686873. (rs)
- O'Donnell, Daniel Paul. 2013. “‘I Certainly Have Subjects in My Mind’: The Diary of Anne Frank as Bildungsroman.” *Canadian Journal of Netherlandic Studies* 32: 49–88. https:/\allowbreak{}/\allowbreak{}doi.org/\allowbreak{}10.5281/\allowbreak{}zenodo.3596110. (r)
- O'Donnell, Daniel Paul. 2012. “Move Over: Learning to Read (and Write) with Novel Technology.” *Scholarly and Research Communication* 3 (4). https:/\allowbreak{}/\allowbreak{}doi.org/\allowbreak{}10.22230/\allowbreak{}src.2012v3n4a68. (r)
- O'Donnell, Daniel Paul. 2010. “Different Strokes, Same Folk: Designing the Multi-form Digital Edition.” *Literature Compass* 7 (2). http:/\allowbreak{}/\allowbreak{}dx.doi.org/\allowbreak{}10.1111/\allowbreak{}j.1741-4113.2009.00683.x. (r)
- O'Donnell, Daniel Paul. 2009. “Byte Me: Technological Education and the Humanities.” *Heroic Age* 12. http:/\allowbreak{}/\allowbreak{}www.mun.ca/\allowbreak{}mst/\allowbreak{}heroicage/\allowbreak{}issues/\allowbreak{}12/\allowbreak{}em.php. (i)
- O'Donnell, Daniel Paul. 2009. “Back to the Future: What Digital Editors Can Learn from Print Editorial Practice.” *Literary and Linguistic Computing* 24: 113–125. https:/\allowbreak{}/\allowbreak{}doi.org/\allowbreak{}10.1093/\allowbreak{}llc%2Ffqn039. (r)
- Lee, Stuart, and Daniel Paul O'Donnell. 2009. “From Manuscript to Computer.” In *Working with Anglo-Saxon Manuscripts*, edited by Gale R. Owen-Crocker, 253–284. Exeter: Exeter UP. (r)
- O'Donnell, Daniel Paul. 2008. “Resisting the Tyranny of the Screen, or, Must a Digital Edition Be Electronic?” *Heroic Age* 11. http:/\allowbreak{}/\allowbreak{}www.heroicage.org/\allowbreak{}issues/\allowbreak{}11/\allowbreak{}em.php. (i)
- Bodard, Gabriel, and Daniel Paul O'Donnell. 2008. “We Are All Together: On Publishing a Digital Classicist Issue of the *Digital Medievalist* Journal.” *Digital Medievalist* 4. http:/\allowbreak{}/\allowbreak{}doi.org/\allowbreak{}10.16995/\allowbreak{}dm.18. (ic)
- O'Donnell, Daniel Paul. 2007. “Disciplinary Impact and Technological Obsolescence in Digital Medieval Studies.” In *A Companion to Digital Literary Studies*, edited by Ray Siemens and Susan Schreibman. Cambridge: Blackwell. https:/\allowbreak{}/\allowbreak{}doi.org/\allowbreak{}10.1002/\allowbreak{}9781405177504.ch3. (r)
- O'Donnell, Daniel Paul. 2007. “Material Differences: The Place of Cædmon's Hymn in the History of Anglo-Saxon Vernacular Poetry.” In *Cædmon's Hymn and Material Culture in the World of Bede*, edited by Allen J. Frantzen and John Hines, 15–50. Morgantown VA: West Virginia University Press. (r)
- O'Donnell, Daniel Paul. 2007. “If I Were ‘You’: How Academics Can Stop Worrying and Learn to Love ‘the Encyclopedia That Anyone Can Edit.’” *Heroic Age* 10. http:/\allowbreak{}/\allowbreak{}www.heroicage.org/\allowbreak{}issues/\allowbreak{}10/\allowbreak{}em.html. (i)
- O'Donnell, Daniel Paul. 2005. “The Ghost in the Machine: Revisiting an Old Model for the Dynamic Generation of Digital Editions.” *HumanIT* 8 (1): 51–71. Offprint: https:/\allowbreak{}/\allowbreak{}zenodo.org/\allowbreak{}record/\allowbreak{}3596125. (i)
- O'Donnell, Daniel Paul. 2005. “O Captain! My Captain! Using Technology to Guide Readers Through an Electronic Edition.” *Heroic Age* 8. http:/\allowbreak{}/\allowbreak{}www.heroicage.org/\allowbreak{}issues/\allowbreak{}8/\allowbreak{}em.html. (i)
- O'Donnell, Daniel Paul. 2004. “The Digital Medievalist Project.” *Old English Newsletter* 37: 19–21. (i)
- O'Donnell, Daniel Paul. 2004. “Bede’s Strategy in Paraphrasing Cædmon’s Hymn.” *JEGP* 103: 417–433. (r)
- O'Donnell, Daniel Paul. 2004. “The Doomsday Machine, or, ‘If You Build It, Will They Still Come Ten Years From Now?’: What Medievalists Working in Digital Media Can Do to Ensure the Longevity of Their Research.” *Heroic Age* 7. http:/\allowbreak{}/\allowbreak{}www.heroicage.org/\allowbreak{}issues/\allowbreak{}7/\allowbreak{}ecolumn.html. (i)
- O'Donnell, Daniel Paul. 2004. “Numeric and Geometric Patterning in Cædmon’s Hymn.” *ANQ* 17: 3–12. (r)
- O'Donnell, Daniel Paul. 2003. “‘Pioneers! O Pioneers!’: Some Electronic Editing Do’s and Don’ts From Stijn Streuvels’s *De Teleurgang van den Waterhoek*.” *Literary and Linguistic Computing*. (r)
- O'Donnell, Daniel Paul. 2002. “Junius’s Knowledge of the Old English Poem *Durham*.” *Anglo-Saxon England* 30: 231–245. (r)
- O'Donnell, Daniel Paul. 2002. “The Accuracy of the St. Petersburg Bede.” *Notes and Queries* 247: 4–6. (r)
- O'Donnell, Daniel Paul. 2001. “Fish and Fowl: Generic Expectations and the Relationship Between the Old English *Phoenix* Poem and Lactantius’s *De Ave Phoenice*.” In *Germanic Texts and Latin Models: Medieval Reconstructions*, edited by K.E. Olsen, A. Harbus, and T. Hofstra. *Germania Latina IV. Mediaevalia Groningana*, n.s. 2. Leuven, Paris and Sterling, VA: Peeters, 157–171. (r)
- O'Donnell, Daniel Paul. 1999. “Hædre and Hædre Gehogode (Solomon and Saturn Line 62b and *Resignation* Line 63a).” *Notes and Queries* 244: 312–316. (r)
- O'Donnell, Daniel Paul. 1998. “The Spirit and the Letter: Literary Embellishment in Old Frisian Legal Texts.” *Amsterdamer Beiträge zur älteren Germanistik* 49: 245–256. (r)
- O'Donnell, Daniel Paul\*. 1996. “A Northumbrian Version of ‘Cædmon’s Hymn’ (eordu-recension) in Brussels Bibliothèque Royale Manuscript 8245–57 ff.62r2–v1: Identification, Edition, and Filiation.” In *Beda Venerabilis: Historian, Monk and Northumbrian*, edited by L.A.R.J. Houwen and A.A. MacDonald, 139–166. Groningen: Egbert Forsten. (r)
- O'Donnell, Daniel Paul\*. 1995. “Schoolbook Design in the Fifteenth Century.” *The Yale University Library Gazette* 70: 23–38. (r)
- O'Donnell, Daniel Paul\*. 1991. “The Collective Sense of Concrete Singular Nouns in *Beowulf*: Emendations of Sense.” *Neuphilologische Mitteilungen* 92: 433–440. (r)

\sectionbreak{}
## Novel Forms of Scholarly Dissemination

- Eve, Martin Paul, Daniel Paul O’Donnell, Cameron Neylon, Samuel Moore, Robert Gadie, Victoria Odeniyi, and Shahina Parvin. 2021. “Reading Peer Review – What a Dataset of Peer Review Reports Can Teach Us about Changing Research Culture.” *Impact of Social Sciences* (blog), March 31. https:/\allowbreak{}/\allowbreak{}blogs.lse.ac.uk/\allowbreak{}impactofsocialsciences/\allowbreak{}2021/\allowbreak{}03/\allowbreak{}31/\allowbreak{}reading-peer-review-what-a-dataset-of-peer-review-reports-can-teach-us-about-changing-research-culture/\allowbreak{}.
- Bosman, Jeroen, Ian Bruno, Chris Chapman, Bastian Greshake Tzovaras, Nate Jacobs, Bianca Kramer, Maryann Martone, Fiona Murphy, Daniel Paul O’Donnell, Michael Bar-Sinai, Stephanie Hagestrom, Josh Utley, and Lusia Veksler [alphabetical order]. 2017. “The Scholarly Commons – Principles and Practices to Guide Research Communication.” *OSF Preprints*. https:/\allowbreak{}/\allowbreak{}doi.org/\allowbreak{}10.17605/\allowbreak{}OSF.IO/\allowbreak{}6C2XT.
- Moore, Samuel\*, Cameron Neylon, Martin Paul Eve, Daniel Paul O’Donnell, and Damian Pattinson [random author order]. 2016. “Excellence R Us: University Research and the Fetishisation of Excellence.” [Preprint]. https:/\allowbreak{}/\allowbreak{}figshare.com/\allowbreak{}articles/\allowbreak{}Excellence_R_Us_University_Research_and_the_Fetishisation_of_Excellence/\allowbreak{}3413821. https:/\allowbreak{}/\allowbreak{}doi.org/\allowbreak{}10.6084/\allowbreak{}m9.figshare.3413821.v1.
\sectionbreak{}
  - **Journalism about this article**:
    - Carmichael, Joe. 2016. “Science Has an Excellence Problem: Why the Incessant Quest for Academic Excellence Leads to Bad Science.” *Inverse*, June 14. http:/\allowbreak{}/\allowbreak{}bit.ly/\allowbreak{}ExcellenceInverse.
    - Matthews, David. 2016. “Focus on Research ‘Excellence’ Is ‘Damaging Science.’” *Times Higher Education Supplement*. 
    https:/\allowbreak{}/\allowbreak{}www.\allowbreak{}timeshighereducation.\allowbreak{}com/\allowbreak{}news/\allowbreak{}focus-on-research-excellence-is-damaging-science.
- O’Donnell, Daniel Paul. 2016. “Synchronic Similarity in Scholarly Communication May Mask Diachronically Distinct Goals and Histories.” *Journal of Brief Ideas*, May 30. https:/\allowbreak{}/\allowbreak{}doi.org/\allowbreak{}10.5281/\allowbreak{}zenodo.53748.
- O’Donnell, Daniel Paul. 2016. “A First Law of Humanities Computing.” *Journal of Brief Ideas*, March 13. https:/\allowbreak{}/\allowbreak{}doi.org/\allowbreak{}10.5281/\allowbreak{}zenodo.47473.
- O’Donnell, Daniel Paul. 2014. “The Credit Line.” *Digital Humanities Now*, Editors’ Choice, July. http:/\allowbreak{}/\allowbreak{}digitalhumanitiesnow.org/\allowbreak{}2014/\allowbreak{}07/\allowbreak{}editors-choice-the-credit-line/\allowbreak{}.
- O’Donnell, Daniel Paul. 2012. “‘There’s No Next about It’: Stanley Fish, William Pannapacker, and the Digital Humanities as Paradiscipline.” *Digital Humanities Now*, Editors’ Choice, September. http:/\allowbreak{}/\allowbreak{}digitalhumanitiesnow.org/\allowbreak{}2012/\allowbreak{}09/\allowbreak{}editors-choice-theres-no-next-about-it-stanley-fish-william-pannapacker-and-the-digital-humanities-as-paradiscipline-dpod-blog/\allowbreak{}.
- Earl, Jim, Pat Conner, Karen Jolly, Sarah Higley, Sarah Keefer, Connie Hieatt, Dan O’Donnell\*, et al. 1990. “Bi-Coastal Beowulfians of the ’90s: A Curious ANSAXNET Conversation [Excerpted from ANSAXNET, December 1990–February 1991].” *Old English Newsletter* 24 (1): 36–39. http:/\allowbreak{}/\allowbreak{}www.oenewsletter.org/\allowbreak{}OEN/\allowbreak{}archive/\allowbreak{}OEN24_1.pdf. [Author order by contribution]

\sectionbreak{}
## Working Papers

- O’Donnell, Daniel Paul. 2016. “The DHSI Analogy: Rationale, Growth, History, and Business Model.” University of California San Diego, September 18.
- Co-Author with the TEI Technical Council (esp. Peter Boot, Daniel Paul O'Donnell, Dot Porter, Gabriel Bodard, Arianna Ciula). 2009. “Response to: Request for Information Regarding the Weekly Notes of Dr. Wernher von Braun (NNH09CAO002L).” NASA (Space Operations Directorate), August 31.
- O'Donnell, Daniel Paul, James Cummings, and Roberto Rosselli Del Turco. 2006. “Why Should I Write for Your Wiki?” Readex Community Academic Advisory Board, April 21.
- O'Donnell, Daniel Paul. 1999. “An Undocumented Method of Filtering and Translating Structural SGML to HTML Using Citec Multidoc Pro Style Sheets.”

\sectionbreak{}
## Reviews and Encyclopædia Entries

- O'Donnell, Daniel Paul. “Cædmon.” *Wikipedia*. Featured article, June 2006–present. 
    *Published on Wikipedia’s front page, July 7, 2006, after I did a complete revision. Approximately one in 10,000 articles receive featured status.*
- O'Donnell, Daniel Paul. 1999. Review of Hal Momma, *The Composition of Old English Poetry* (Cambridge Studies in Anglo-Saxon England, 20). *Early Medieval Europe* 8.1: 163–166. (i)
- O'Donnell, Daniel Paul. 2002. Review of Timothy Graham, ed., *The Recovery of Old English: Anglo-Saxon Studies in the Sixteenth and Seventeenth Centuries* (Publications of the Richard Rawlinson Center). *Early Medieval Europe* 11.1: 93–95. (i)
- O'Donnell, Daniel Paul. 2002. Review of Donald Scragg and Carole Weinberg, eds., *Literary Appropriations of the Anglo-Saxons from the Thirteenth to the Twentieth Centuries* (Cambridge Studies in Anglo-Saxon England, 29). *Early Medieval Europe* 11.2. (i)
- O'Donnell, Daniel Paul. 2008. Review of Thomas A. Bredehoft, *Early English Metre*. Toronto Old English Series. University of Toronto Press. *Heroic Age* 10. (i)
- O'Donnell, Daniel Paul. 2009. Review of Martin Foys, *Virtually Anglo-Saxon*. Florida University Press. *Review of English Studies* 60: 475–476. (i)
- O'Donnell, Daniel Paul. 2011. Review of Jun Terasawa, *Old English Metre: An Introduction*. Toronto Anglo-Saxon Series. Toronto: University of Toronto Press. *The Medieval Review* 2011-09-28. https:/\allowbreak{}/\allowbreak{}scholarworks.iu.edu/\allowbreak{}dspace/\allowbreak{}bitstream/\allowbreak{}handle/\allowbreak{}2022/\allowbreak{}13595/\allowbreak{}11.09.28.html?sequence=1 (i)
- O'Donnell, Daniel Paul. 2015. Review of *A Conspectus of Scribal Hands Writing in English, 960–1100*. *Journal of English and Germanic Philology* 114.2: 294–297. (i)
- Copland, Colleen\*, Stephen Carrell\*, Gwendolyn Davidson\*, Virgil Grandfield\*, and Daniel Paul O'Donnell. 2016. Review of *Electronic Beowulf*. *Digital Medievalist* 10. https:/\allowbreak{}/\allowbreak{}doi.org/\allowbreak{}10.16995/\allowbreak{}dm.56. (rsc)

\sectionbreak{}
# Knowledge Mobilisation
**Summary**: Through public journalism, podcast interviews, and media commentary, O'Donnell engages broader audiences on issues related to academic freedom, digital infrastructure, and postsecondary governance.

\sectionbreak{}
## Scholarly Interviews with Me

- Eve, Martin Paul, Daniel Paul O’Donnell, Robert Gadie\*, Victoria Odeniyi\*, and Shahina Parvin\*. 2021. “Reading Peer Review: PLOS One and…” Podcast interview. *New Books Network*, June 1. https:/\allowbreak{}/\allowbreak{}newbooksnetwork.com/\allowbreak{}reading-peer-review
- Cook, Eleanor I., and Daniel Paul O’Donnell. 2019. “An Interview with Daniel O’Donnell, Current President of FORCE11.” *The Serials Librarian* 76.1–2: 1–3. https:/\allowbreak{}/\allowbreak{}doi.org/\allowbreak{}10.1080/\allowbreak{}0361526X.2019.1643437 (ri)
- Priego, Ernesto, and Daniel Paul O’Donnell. 2013. “Bringing Diversity of Experience Together: An Interview with Daniel O’Donnell.” *4Humanities*. Accessed May 9. http:/\allowbreak{}/\allowbreak{}4humanities.org/\allowbreak{}2013/\allowbreak{}05/\allowbreak{}interview-daniel-o-donnell/\allowbreak{}.
  - Also translated into Spanish: http:/\allowbreak{}/\allowbreak{}4humanities.org/\allowbreak{}2013/\allowbreak{}06/\allowbreak{}diversidad-y-experiencia-una-entrevista-con-daniel-odonnell/\allowbreak{}
  - And Japanese: http:/\allowbreak{}/\allowbreak{}www.jadh.org/\allowbreak{}godh (i)

\sectionbreak{}
## Journalism by Me

- O’Donnell, Daniel Paul. 2024. “Communication and Leadership a Matter of Trust.” *Lethbridge Herald*, September 20. https:/\allowbreak{}/\allowbreak{}lethbridgeherald.com/\allowbreak{}commentary/\allowbreak{}opinions/\allowbreak{}2024/\allowbreak{}09/\allowbreak{}20/\allowbreak{}communication-and-leadership-a-matter-of-trust/\allowbreak{}
- O’Donnell, Daniel Paul. 2024. “A Tale of Two Edmontons Seen at Awards.” *Lethbridge Herald*, October 4. https:/\allowbreak{}/\allowbreak{}lethbridgeherald.com/\allowbreak{}commentary/\allowbreak{}opinions/\allowbreak{}2024/\allowbreak{}10/\allowbreak{}04/\allowbreak{}a-tale-of-two-edmontons-seen-at-awards/\allowbreak{}
- O’Donnell, Daniel Paul. 2024. “Has the University of Lethbridge Lost Its Soul?” *Lethbridge Herald*, September 6. https:/\allowbreak{}/\allowbreak{}lethbridgeherald.com/\allowbreak{}commentary/\allowbreak{}opinions/\allowbreak{}2024/\allowbreak{}09/\allowbreak{}06/\allowbreak{}has-the-university-of-lethbridge-lost-its-soul/\allowbreak{}
- O’Donnell, Daniel Paul. 2023. “Reflections on the U of L a Year after Strike/\allowbreak{}Lockout.” *Lethbridge Herald*, March 21. https:/\allowbreak{}/\allowbreak{}lethbridgeherald.com/\allowbreak{}commentary/\allowbreak{}opinions/\allowbreak{}2023/\allowbreak{}03/\allowbreak{}21/\allowbreak{}reflections-on-the-u-of-l-a-year-after-strike-lockout/\allowbreak{}
- O’Donnell, Daniel Paul. 2022. “When the Ivy Tower Is a Fishing Hole.” *Lethbridge Herald*, January 12. https:/\allowbreak{}/\allowbreak{}lethbridgeherald.com/\allowbreak{}commentary/\allowbreak{}opinions/\allowbreak{}2022/\allowbreak{}01/\allowbreak{}12/\allowbreak{}when-the-ivy-tower-is-a-fishing-hole/\allowbreak{}
- O’Donnell, Daniel Paul. 2022. “Maybe We Need to Start Demonstrating Again.” *Lethbridge Herald*, January 29. https:/\allowbreak{}/\allowbreak{}lethbridgeherald.com/\allowbreak{}commentary/\allowbreak{}opinions/\allowbreak{}2021/\allowbreak{}12/\allowbreak{}08/\allowbreak{}maybe-we-need-to-start-demonstrating-again/\allowbreak{}
- O’Donnell, Daniel Paul. 2021. “There Is Still Time to Turn Things Around at the U of L.” *Lethbridge Herald*, September 10. https:/\allowbreak{}/\allowbreak{}lethbridgeherald.com/\allowbreak{}commentary/\allowbreak{}opinions/\allowbreak{}2021/\allowbreak{}09/\allowbreak{}10/\allowbreak{}there-is-still-time-to-turn-things-around-at-u-of-l/\allowbreak{}
- O’Donnell, Daniel Paul. 2016. “Customized Pronouns: A Good Idea That Makes No Sense.” *The Globe and Mail*, October 15. https:/\allowbreak{}/\allowbreak{}www.theglobeandmail.com/\allowbreak{}opinion/\allowbreak{}customized-pronouns-a-good-idea-that-makes-no-sense/\allowbreak{}article32373933/\allowbreak{}
- O’Donnell, Daniel Paul. 2012. “Why Isn’t the Internet Obsolete?” *Lethbridge Herald*, February 18. http:/\allowbreak{}/\allowbreak{}www.lethbridgeherald.com/\allowbreak{}public-professor/\allowbreak{}why-isnt-the-internet-obsolete-21812.html
- O’Donnell, Daniel Paul. 2010. “Gun-list Debate Way Off Target.” *The Globe and Mail*, September 14, A15. http:/\allowbreak{}/\allowbreak{}bit.ly/\allowbreak{}cxjGGZ
- O’Donnell, Daniel Paul. 2010. “The Copy-and-Paste Generation.” *National Post*, September 7, A15. http:/\allowbreak{}/\allowbreak{}bit.ly/\allowbreak{}bMl5Nw
- O’Donnell, Daniel Paul. 2010. “Humanities, Not Science, Key to New Web Frontier.” *Edmonton Journal*, July 21. http:/\allowbreak{}/\allowbreak{}bit.ly/\allowbreak{}aApodN
- O’Donnell, Daniel Paul. 2004. “More Research Money Needed for Social Science & Humanities.” *CBC Radio One*, March 15. Transcript: http:/\allowbreak{}/\allowbreak{}bit.ly/\allowbreak{}akgKs9
- O’Donnell, Daniel Paul. 1998. “Restoring Pages to a Sacred Text [The Diary of Anne Frank].” *The Globe and Mail*, November 17, Arts and Leisure, A17–A18.

\sectionbreak{}
## Journalism and Interviews About Me and My Work

- Schmidt, Scott, Jeremy Appel, Mo Cranker, and Daniel O’Donnell. 2022. “ULFA on Strike with President Dan O’Donnell.” Podcast interview. *The Forgotten Corner*, Episode 63, March 17. https:/\allowbreak{}/\allowbreak{}www.forgottencornerpod.com/\allowbreak{}episodes/\allowbreak{}episode-63-ulfa-on-strike-with-president-dan-odonnell
- Srivastava, Sanjay, Alexa Tullett, and Simine Vazire. 2017. “Excellence Adventures.” *The Black Goat*, April 19. https:/\allowbreak{}/\allowbreak{}www.theblackgoatpodcast.com/\allowbreak{}posts/\allowbreak{}excellence-adventures/\allowbreak{}
- UNews. 2015. “U of L Researchers Benefit from Canada Foundation for Innovation Funding.” July 29. http:/\allowbreak{}/\allowbreak{}www.uleth.ca/\allowbreak{}unews/\allowbreak{}article/\allowbreak{}u-l-researchers-benefit-canada-foundation-innovation-funding
- Carmichael, Joe. 2016. “Science Has an Excellence Problem.” *Inverse*, June 14. http:/\allowbreak{}/\allowbreak{}bit.ly/\allowbreak{}ExcellenceInverse
- Matthews, David. 2016. “Focus on Research ‘Excellence’ Is ‘Damaging Science.’” *Times Higher Education Supplement*. https:/\allowbreak{}/\allowbreak{}www.timeshighereducation.com/\allowbreak{}news/\allowbreak{}focus-on-research-excellence-is-damaging-science
- Cooney, Bob. 2012. “O’Donnell, Graham Bring 3D Imaging to Ancient Cross.” *UNews*, June 18. http:/\allowbreak{}/\allowbreak{}www.uleth.ca/\allowbreak{}unews/\allowbreak{}article/\allowbreak{}odonnell-graham-bring-3d-imaging-ancient-cross
- “Centuries-Old Tale Gets Modern Twist.” *Dumfries and Galloway Standard*, April 20, 2012, p. 9.
- Interview about “The Copy-and-Paste Generation” (*National Post*) on Shaw TV, Lethbridge, September 22, 2010.
- Interview about “Gun-list Debate Way Off Target” (*Globe and Mail*) on CBC Radio One’s *Cross Country Checkup*, September 19, 2010. http:/\allowbreak{}/\allowbreak{}bit.ly/\allowbreak{}cFsK8B Podcast: http:/\allowbreak{}/\allowbreak{}bit.ly/\allowbreak{}d1EN3h (at ~32')
- Interview about “Gun-list Debate Way Off Target” (*Globe and Mail*) on CBC Radio One (Winnipeg), *Information Radio*, September 15, 2010.
- Interview about “The Copy-and-Paste Generation” (*National Post*) on CFAX 1070, September 8, 2010.
- Interview on academic use of Wikipedia on CBC Radio One (British Columbia), *BC Almanac*, March 3, 2008.
- Interview on academic use of Wikipedia on CBC Radio One (Alberta), *Wild Rose Country*, February 12, 2008.
- Interview about *Cædmon’s Hymn* and TEI work on CBC Radio One (Alberta), *Wild Rose Country*, December 21, 2006.
- “Iraq War Coverage More Balanced than 1991: Prof.” *Lethbridge Herald*, April 3, 2003.

\sectionbreak{}
## Creative Work

- **Author**, *Glamour Boys /\allowbreak{} Once We Were Heroes*. A 105,000-word upmarket literary noir novel set primarily in 1960s Toronto. The work engages themes of post-war trauma, class, and self-deception through the investigation of war crimes and personal memory.
- **Actor**, *Homo*, a Latin play produced by the medieval drama troupe *Poculi Ludisque Societas* (Centre for Medieval Studies, University of Toronto); performed at the International Association for Neo-Latin Studies Conference, Toronto, 1988.
- **Editor, Producer, and Technician**, *Within the Hollow Crown: A Sad Story of the Death of Kings*. A radio adaptation of William Shakespeare’s *Richard II* and *Henry IV* parts 1 and 2. Adapted and produced by Dan O’Donnell. Directed by Ralph Joneikeiss. Broadcast on CIUT-FM (Toronto), March 22 and April 5, 1987. [http:/\allowbreak{}/\allowbreak{}bit.ly/\allowbreak{}a6wqIh](http:/\allowbreak{}/\allowbreak{}bit.ly/\allowbreak{}a6wqIh)

\sectionbreak{}
# Lectures and Conferences
**Summary**: O'Donnell has presented and organised sessions at national and international conferences across multiple disciplines, and regularly teaches and lectures at summer institutes such as DHSI and FSCI. He has given over 125 lectures to national and international audiences.

\sectionbreak{}
## Institutes and conferences organised

\sectionbreak{}
### Summer Schools
- Force11 Scholarly Communications Institute 2024. Steering Committee (Member), Program Committee, Local Organizing Committee, Code of Conduct Committee (Member). UCLA hybrid, August 2024. http:/\allowbreak{}/\allowbreak{}force11.org/\allowbreak{}fsci.
- Force11 Scholarly Communications Institute 2023. Steering Committee (Member), Program Committee, Local Organizing Committee, Code of Conduct Committee (Member). UCLA hybrid, August 2023. http:/\allowbreak{}/\allowbreak{}force11.org/\allowbreak{}fsci.
- Force11 Scholarly Communications Institute 2022. Steering Committee (Member), Program Committee, Local Organizing Committee, Code of Conduct Committee (Member). UCLA online, August 2022. http:/\allowbreak{}/\allowbreak{}force11.org/\allowbreak{}fsci.
- Force11 Scholarly Communications Institute 2021. Steering Committee (Member), Program Committee, Local Organizing Committee, Code of Conduct Committee (Member). UCLA online, August 2021. http:/\allowbreak{}/\allowbreak{}force11.org/\allowbreak{}fsci.
- Force11 Scholarly Communications Institute 2020. Steering Committee (chair), Program Committee, Local Organizing Committee, Code of Conduct Committee (chair). UCLA online, August 2020. http:/\allowbreak{}/\allowbreak{}force11.org/\allowbreak{}fsci.
- Force11 Scholarly Communications Institute 2019. Steering Committee (chair), Program Committee, Local Organizing Committee, Code of Conduct Committee (chair). UCLA, August 2019.
- Force11 Scholarly Communications Institute 2018. Steering Committee (chair), Program Committee, Local Organizing Committee, Code of Conduct Committee (co-chair). UC San Diego, July-August 2018.
- Force11 Scholarly Communications Institute 2017. Steering Committee (chair), Program Committee, Local Organizing Committee, Code of Conduct Committee (co-chair). UC San Diego, July-August 2018.

\sectionbreak{}
### Conferences, Conference Sessions, and Workshops Sessions Organised (selected)

- Open and Inclusive Access to Research (OIAR). 2021. Co-organiser with Gimena del Rio and Wouter Schallier. Nov. 8-11. Online. SSHRC Connections. http:/\allowbreak{}/\allowbreak{}openandinclusiveaccesstoresearch.org/\allowbreak{}
- Co-designer and co-instructor. Diversity Workshop WIDH 2019. Led by Barbara Bordalejo. IIT Gandhinagar, India. Dec. 2019.
- Co-designer and instructor. Global South. FSCI 2018. Los Angeles. August 3-8, 2018. With Gimena del Rio, Idowu Adegbilero-Iwari, and Samir Hachani.
- Co-designer and co-instructor. Diversity Workshop DH 2019. Led by Barbara Bordalejo. Utrecht, June 2019.
- Organising Committee. Force 2018 Montreal. October 11-12, 2018.
- Organising committee chair. Force 11 Scholarly Communications Institute. San Diego. July 30-August 3, 2018.
- Co-designer and co-instructor. Collaboration and the Scholarly Commons. FSCI 2018. San Diego. July 30-August 03, 2018. With Victoria Antonova, Maryann Martone, and Sergey Parinov
- Co-designer and co-instructor. Diversity Workshop DH 2018. Led by Barbara Bordalejo. Mexico City, June 2018.
- Co-designer and co-instructor. Open South: The Open science experience in Latin America and the Caribbena. FSCI 2018. San Diego. July 30-August 03, 2018. With Gimena Del Rio, April Hathcock, Wouter Schallier.
- Co-designer and co-instructor. Walking the walk: Promoting and maintaining best practice in Fair and Open Evaluation. FSCI 2018. San Diego. July 30-August 03, 2018. With David De Roure, Allegra Swift, and Stefan Tanaka
- Co-designer and instructor. Global South. FSCI 2017. San Diego. July 29-August 2, 2017. With Robin Champieux, and Gimena del Rio.
- Organising committee chair. Force 11 Scholarly Communications Institute. San Diego. July 29-August 2, 2017.
- Designer and instructor. The Scholarly Commons: Principle and Practices in Digital Scholarship.  1st Lagos Summer School in Digital Humanities. University of Lagos, Nigeria. July 12 and 14.
- Organizer, designer, instructor. Negotiation skills workshop. 6 week (18 hour) negotiation skills workshop. University of Lethbridge. February 21-March 28, 2017.
- Organizer and coordinator. “Understanding the scholarly commons [Workshop].” KULeuven. September 13-14, 2016.
- Organizer. “PreH34: Translation Hack-a-thon!: Applying the Translation Toolkit to a Global dh+lib.” DH 2016. Krakow. July 12, 2016.
- Organizer. “Boundary Land: Diversity as a defining feature of the Digital Humanities.” July 13, 2016. Dh 2016. Krakow.
- Organizer. “Future Commons.” Invited Workshop. Force 11, Madrid. February 26-27, 2016.
- Organizer. “Digital Cultural Heritage.” Paper session. Congress 2015. Ottawa. June 2, 2015.
- Organizer. “Using Technology and Grading to Create a Scholarly Ecosystem in the Classrom: A workshop.” Paper Session. Spark. University of Lethbridge. April 30, 2015.
- Organizing Committee, Force 2015. Oxford University. January 10-14, 2015.
- “Kickstarting the GO::DH minimal computing working group.” Workshop. DH 2014 Lausanne. July 8, 2014. Leadership: John Edward Simpson, Jentery Sayers, Daniel Paul O'Donnell, Alex Gil.
- “Global outlook::digital humanities: promoting digital humanities research across disciplines, regions, and cultures.” Paper session. DH 2014 Lausanne. 9 July, 2014. Leadership: Barbara Bordalejo, Alex Gil, Roopika Risam, Paul Spence, Elena González-Blanco.
- “Text Encoding.” 2° Encuentro de humanistas digitales, México. 20 May, 2014.
- Organising Committee. 2° Encuentro de humanistas digitales, México. 21-23 May, 2014.
- Organising Committee. DH 2014 Lausanne. Conference Organising Committee. Lausanne Switzerland. 2014.
- “Immersive interpretation and the small cultural heritage site: the case of Ruthwell Kirk.” Session. Canadian Society for Digital Humanities. Victoria. 5 June, 2013.
- “Digital Humanities in Africa: The Case of Nigeria.” Session. Canadian Society for Digital Humanities. Victoria. 3 June, 2013.
- “Global Outlook:: Digital Humanities.” Unconference workshop, Havana. 13 December, 2012. Leadership: Daniel Paul O'Donnell, Alex Gil, Neil Fraistat, Ray Siemens.
- Organising Committee. DH 2012 Hamburg. Conference Organising Committee. Hamburg. July-August, 2012.
- Organising Committee. CSDH-SCHN Conference. Waterloo. June 2012.
- Organising Committee. TEI Members Meeting and Conference (Wurzburg, Germany). November 2011.
- Organising Committee. TEI Members Meeting and Conference (University of Zadar, Croatia). November 2010.
- Organising Committee. TEI Members Meeting and Conference (University of Michigan, Ann Arbor). November 2009.
- Organising Committee. Digital Medievalist Day, ENS-Lyons. April 2009.
- Organising Committee. TEI Members Meeting and Conference (King's College London). November 2008.
- Organising Committee. CASTA. University of Saskatchewan. October 2008.
- Organising Committee. TEI Members Meeting and Conference (University of Maryland). November 2008.

\sectionbreak{}
## Research Lectures

- Daniel Paul O’Donnell. 2024. “Academic Freedom, Privilege, and Intersectionality.” July 29.  Force11 Scholarly Communications Institute (FSCI), Los Angeles. Zenodo. https:/\allowbreak{}/\allowbreak{}doi.org/\allowbreak{}10.5281/\allowbreak{}zenodo.13121345 (i)
- Bordalejo, Barbara, Davide Pafumi\*, Frank Onuh\*, AKM Iftekhar Khalid\*, and Daniel O’Donnell. 2024. “Scarlet Cloak and the Forest Adventure: The Issue of False Positives in AI Detection Tools.” June 17. Congress of the Humanities and Social Sciences (CSDH/\allowbreak{}SCHN). Montreal. https:/\allowbreak{}/\allowbreak{}doi.org/\allowbreak{}10.5281/\allowbreak{}zenodo.12011536. (rs)
- Bordalejo, Barbara, Davide Pafumi, Morgan Slayde Pearce\*, and Daniel O’Donnell. 2024. “Adapting a Research Tool for Teaching in a Post-Pandemic World.” June 17. Congress of the Humanities and Social Sciences (CSDH/\allowbreak{}SCHN). Montreal.  https:/\allowbreak{}/\allowbreak{}doi.org/\allowbreak{}10.5281/\allowbreak{}zenodo.12018101. (rs)
- O’Donnell, Daniel Paul, Barbara Bordalejo, and Nathan D. Woods. 2024. “Data in/\allowbreak{}And Mediaeval Studies.” International Congress on Medieval Studies. Kalamazoo. May 11. https:/\allowbreak{}/\allowbreak{}doi.org/\allowbreak{}10.5281/\allowbreak{}zenodo.11179271. (r)
- Woods, Nathan, Barbara Bordalejo, and Daniel O’Donnell. 2023. “Data Problems in the Humanities, or ‘When Everybody Is Special, No One Is’?” October 3. https:/\allowbreak{}/\allowbreak{}doi.org/\allowbreak{}10.5281/\allowbreak{}zenodo.8403789. (r)
- Bordalejo, Barbara, Daniel Paul O’Donnell, and Nathan Woods. 2023. “The Implications of Multiple Hierarchies for the Future of Humanities Data: Or, What Is a Markup Language, Actually?” October 5. https:/\allowbreak{}/\allowbreak{}doi.org/\allowbreak{}10.5281/\allowbreak{}zenodo.8411165. (r)
- O’Donnell, Bordalejo, and Woods. 2023. “Representational Data:  A Case Study.” May 30.  Congress of the Social Sciences and Humanities. https:/\allowbreak{}/\allowbreak{}doi.org/\allowbreak{}10.5281/\allowbreak{}zenodo.7986430. (r)
- Daniel Paul O’Donnell, Barbara Bordalejo, Nathan Woods, Roberto Rosselli Del Turco. 2022. “Small Data projects/\allowbreak{}Big Data research: contemporary problems and historical solutions.” DH 2022 (Tokyo/\allowbreak{}online). July 28. https:/\allowbreak{}/\allowbreak{}doi.org/\allowbreak{}10.5281/\allowbreak{}zenodo.6857202. (r)
- Daniel Paul O’Donnell. 2022. “‘Good things come in small packets’: Investigating Humanities Research Data Practices: A Community of Practice Approach.” FSCI 2022. Lightning Talks. July 25-29. https:/\allowbreak{}/\allowbreak{}doi.org/\allowbreak{}10.5281/\allowbreak{}zenodo.6902821.
- Daniel Paul O’Donnell. 2022. “REPO: Reimagining Educational Practices for Open 2020-2021.” FSCI 2022. Force11 Working Group Bazaar. July 25. https:/\allowbreak{}/\allowbreak{}doi.org/\allowbreak{}10.5281/\allowbreak{}zenodo.6902861. (i)
- Daniel Paul O’Donnell, Barbara Bordalejo, Nathan Woods, Roberto Rosselli Del Turco. 2022. “Small Data Management in Humanities and Cultural Heritage Projects.” DH Unbound. May 17. https:/\allowbreak{}/\allowbreak{}doi.org/\allowbreak{}10.5281/\allowbreak{}zenodo.6773410. (r)
- Daniel Paul O’Donnell. 2022. “Thinking about CARE principles in the Digital Humanities.  Why CARE may not be only a matter for researchers working with indigenous peoples.” RDA-DU. Online. Feb 21. (i)
- Daniel P. O'Donnell. 2021. ''Thinking about CARE principles in the Digital Humanities. Why CARE may not be only a matter for researchers working with indigenous peoples.” DARIAH Friday Frontiers. Online. October. France. October 8. (i)
- Daniel P. O'Donnell, Rosselli Del Turco R. 2020. 'Methods and tools to build a web-based edition on the basis of FAIR/\allowbreak{}OA data'. Assemblée Générale du consortium Cahier. France. November 27. https:/\allowbreak{}/\allowbreak{}cahier.hypotheses.org/\allowbreak{}5384. (i)
- O'Donnell, Daniel. (2020, August). How to make a manifesto… and not: Principles of the Scholarly Commons as Case History (Version Presentation). Presented at the Force11 Scholarly Communication Institute 2020 Online (FSCI 2020 Online), Online: Zenodo. http:/\allowbreak{}/\allowbreak{}doi.org/\allowbreak{}10.5281/\allowbreak{}zenodo.3979532. (i)
- Bordalejo, Barbara, Folgert Karsdorp, Daniel ODonnell, Basten Stokhuyzen, and Karina Van Dalen-Oskam. 2020. “Check Your Privilege: The Digital Privilege Game.” Congress 2020 (online). June 5. Published in Building Community Online. https:/\allowbreak{}/\allowbreak{}hcommons.org/\allowbreak{}deposits/\allowbreak{}item/\allowbreak{}hc:30179. (r)
- O’Donnell, Daniel, and Virgil Grandfield. 2020. “The Lethbridge Journal Incubator: A Collaborative Student-Expert Open Access Publishing Model.” Open Publishing Fest. May 28. https:/\allowbreak{}/\allowbreak{}doi.org/\allowbreak{}10.5281/\allowbreak{}zenodo.3863022. (i)
- O’Donnell, Daniel Paul. 2019. Publishing (and Forgetting) the Small or Medium-sized Scholarly Edition or Cultural Heritage Collection as Linked Open Data: Using Zenodo and Github to Publish the Visionary Cross Project. Winter Institute in Digital Humanities. IIT Gandhinagar. Palaj, India. December 20. https:/\allowbreak{}/\allowbreak{}zenodo.org/\allowbreak{}record/\allowbreak{}3586007. (Updated version of DH2019 presentation). (i)
- O’Donnell, Daniel Paul. 2019. Small, thick, and slow: Towards an Open and FAIR data culture in the Humanities. Winter Institute in Digital Humanities. IIT Gandhinagar. Palaj, India. December 18. https:/\allowbreak{}/\allowbreak{}zenodo.org/\allowbreak{}record/\allowbreak{}3581520. (i)
- O’Donnell, Daniel Paul. 2019. Publishing (and Forgetting) the Small or Medium-sized Scholarly Edition or Cultural Heritage Collection as Linked Open Data: Using Zenodo and Github to Publish the Visionary Cross Project. Digital Humanities 2019. Utrecht, Netherlands. July 10. https:/\allowbreak{}/\allowbreak{}zenodo.org/\allowbreak{}record/\allowbreak{}3586007. (r)
- O’Donnell, Daniel Paul. 2019. Small, thick, and slow: Thinking about data and research publication in the Humanities in the age of Open and FAIR. Curtin University. Perth, Australia. November 25, 2019. https:/\allowbreak{}/\allowbreak{}zenodo.org/\allowbreak{}record/\allowbreak{}3554760. (i)
- O’Donnell, Daniel Paul. 2018. The Nature of Humanities Data. Open Science Infrastructures for Big Cultural Data: International Advanced Masterclass. Plovdiv, Bulgaria. December 14. http:/\allowbreak{}/\allowbreak{}doi.org/\allowbreak{}10.5281/\allowbreak{}zenodo.2246390. (i)
- O’Donnell, Daniel Paul. 2018. The Humanities and New Technologies: Exploring Digital Tools for Innovation and Development. Keynote address. 2nd Lagos Summer School in Digital Humanities. University of Lagos, Nigeria. October 2. https:/\allowbreak{}/\allowbreak{}doi.org/\allowbreak{}10.5281/\allowbreak{}zenodo.1443255. (i)
- O’Donnell, Daniel Paul. 2018. Open Access/\allowbreak{}Open Science and the Humanities on a Cross-Regional Basis. Invited lecture. Elizade University, Ilara-Mokin, Nigeria. Sept. 28. https:/\allowbreak{}/\allowbreak{}doi.org/\allowbreak{}10.5281/\allowbreak{}zenodo.1455735. (i)
- O’Donnell, Daniel Paul. 2018. Sometimes a poet is just a poet: Reading Bede reading Cædmon’s inspiration in light of Germanic Tradition and cultural analogues. Invited lecture. XIX Seminario Avanzato in Filologia Germanica "Sogni, Visioni e Profezie nella letteratura Germanica medievale. Università degli studi di Torino. Sept. 18. http:/\allowbreak{}/\allowbreak{}doi.org/\allowbreak{}10.5281/\allowbreak{}zenodo.2222396. (i)
- O’Donnell, Daniel Paul. 2018. Open in the Digital Humanities. Some thoughts on cross-disciplinary similarity and difference. DH 2018. Mexico City. (r)
- O’Donnell, Daniel Paul. 2018. Le goût de la mer: Teaching Research. Invited lecture. University of Saskatchewan. May 8. (i)
- Bar-Sinai, Michael, Jeroen Bosman, Ian Bruno, Chris Chapman, Bastian Greshake Tzovaras, Stephanie Hagstrom, Nate Jacobs\*, Bianca Kramer, Maryann Martone, Fiona Murphy, Daniel Paul O'Donnell [alphabetical order]. 2018. “Hier stehe ich! Operationalising conviction in the Scholarly Commons.” McGill. April 4. (isc)
- O'Donnell, Daniel Paul 2017. 'How to Start an Open Access Journal'. Open Access Week. Lethbridge. Canada. (i)
- Heather Bliss\*, Inge Genee, Marie Odile Junker & Daniel Paul O’Donnell, 2017. “Credit where credit is due”: How to handle attribution, copyright, and intellectual property in on-line digital resources for Algonquian languages.” The Algonguin Conference, Université de Montréal. October 28. (rs)
- O’Donnell, Daniel Paul. 2017. Humanities in the Age of Technology [Keynote]. 1st Lagos Summer School in Digital Humanities. University of Lagos, Nigeria. July 11. (i)
- Esau\*, Paul, Carey Viejou\*, Sylvia Chow\*, Jarret McKinnon\*, Reed Parsons\*, and Daniel Paul O'Donnell. 2017. First World Problems: Publishing a Graduate Student Journal in the Anglophone Global North Using Minimal Computing Techniques. Session 9a: Open Access/\allowbreak{}Social Scholarship. Canadian Society for Digital Humanities/\allowbreak{}Société canadienne des humanités numériques. Congress 2017. Ryerson University, Toronto. May 31. (rs)
- O’Donnell. Daniel Paul, Dot Porter, Roberto Rosselli Del Turco, Gurpreet Singh\*. 2017. Let’s Get Nekkid! Stripping the User Experience to the Bare Essentials. Sesion 5a: Digital Editions. Canadian Society for Digital Humanities/\allowbreak{}Société canadienne des humanités numériques. Congress 2017. Ryerson University, Toronto. May 30. (rsc)
- rs Esau\*, Paul, Carey Viejou\*, and Daniel Paul O’Donnell. 2017. Publishing the Unpublishable: The Story of a Graduate Journal and a University that Wouldn’t Submit to it. Canadian Association of Learned Journals. Congress 2017. Ryerson University, Toronto. May 27.
- O’Donnell, Daniel Paul. 2017. “The Bird in Hand: Humanities research in the age of open data.” Research Data Management Interdisciplinary Panel Discussion. Centre for the Study of Scholarly Communication. University Library. University of Lethbridge. March 16. (i)
- O’Donnell, Daniel Paul. 2017. “When is infrastructure really a project?.” Open Scientific Infrastructure: in Search of a Development Model. Gaidar Forum: Russia and the World: The Choice of Priorities. Russian Presidential Academy of National Economy and Public Administration. Moscow. January 14. (i)
- O’Donnell, Daniel Paul. 2016. “Length and Breadth: Why diversity is a core intellectual value in the Digital Humanities.” Intersectionality in Digital Humanities Conference. KU Leuven. September 15. (i)
- O’Donnell, Daniel Paul. 2016. All along the Watchtower: An interdisciplinary approach to understanding the importance technical, disciplinary, and interpersonal diversity within the Digital Humanities. DH 2016. Krakow. July 12. (r)
- O’Donnell, Daniel Paul and Shelaigh Brantford. 2016. “The tip of the iceberg: Transparency and diversity in contemporary DH.” Congress 2016. University of Calgary. June 1. Summarised by Geoffrey Rockwell at http:/\allowbreak{}/\allowbreak{}philosophi.ca/\allowbreak{}pmwiki.php/\allowbreak{}Main/\allowbreak{}CSDH-CGSA2016. (rs)
- Bosman, Jeroen, Ian Bruno, Amy Buckland, Sarah Callaghan, Robin Champieux, Chris Chapman, Stephanie Hagstrom, Bianca Kramer, MaryAnn Martone, Daniel Paul O'Donnell. 2016. Scholarly Commons Working Group Webinar. Force11.org. May 24. https:/\allowbreak{}/\allowbreak{}www.force11.org/\allowbreak{}group/\allowbreak{}SCWG/\allowbreak{}May24webina (r)
- O’Donnell, Daniel Paul. 2016. “All Growed Up: Preparing Gold Open Access Journals for a Post-Incubator Life. Some lessons on transitions and new beginnings.” OA@UNT. University of North Texas. May 20. (i)
- O’Donnell. Daniel Paul. 2016. The Visionary Cross Project and the Lethbridge Journal Incubator. Lethbridge SSHRC Showcase. April 29, 2016. (i)
- O’Donnell, Daniel Paul. 2016. Est ce qu’il y a de hors-edition? or, Can you edit everything? MLA 2016 Austin. Session 215 Editing Unruly Objects. January 8. (ir)
- Moore\*, Samuel, Damian Pattison, Cameron Naylon, Daniel O'Donnell. 2015. The Quality of Qualities. Excellence and Soundness in Scholarly Communication. Mellon Triangle Scholarly Communication Workshop. Durham/\allowbreak{}Raleigh. UNC. October 12-16. (rs)
- Ortega, Élika, Alex Gil, Daniel Paul O'Donnell. 2015. “Psst! An Informal Approach to Expanding the Linguistic Range of the Digital Humanities.” Digital Humanities 2015. Melbourne. July 1. (r)
- O'Donnell, Daniel Paul, Gurpreet Singh\*, Rachel Hanks\*, Roberto Rosselli Del Turco. 2015. “The Old Familiar Faces: On the Consumption of Digital Scholarship.” Digital Humanities 2015. Melbourne. July 2. (r)
- O'Donnell, Daniel Paul. 2015. “Est ce qu’il y a de hors-edition? or, Can you edit everything?” Canadian Society for Digital Humanities. Congress of the Humanities and Social Sciences. Ottawa. June 3. (r)
- O'Donnell, Daniel Paul, Rachel Hanks\*, Gurpreet Singh\*, Roberto Rosselli Del Turco. 2015. “The Old Familiar Faces: On the consumption of (digital) textual scholarship.” Highway 2 Conference, University of Lethbridge May 9.
- O’Donnell, Daniel Paul. 2015. “The 'Unessay': A new approach to not teaching composition.” Spark. University of Lethbridge. April 30. (r)
- O’Donnell, Daniel Paul. 2015. ““If 'if's and 'and's were pots and pans...”: Aligning Open Access Publication with the Research and Teaching Missions of the Public University: The Case of the Lethbridge Journal Incubator.” Advancing Research Communication and Scholarship. Philadelphia. April 15. (i)
- O’Donnell, Daniel Paul. 2015. “‘Living Out Loud: Public Humanities and an Eight Century Cross.” North Carolina State University. April 15. (i)
- O’Donnell, Daniel Paul. 2015. “Critical Mass: Online Communities and the Advancement of Research Communication.” Maynooth University (Ireland). March 5. (i)
- O’Donnell, Daniel Paul. 2015. “‘Living Out Loud: The Visionary Cross Project and the Public Humanities.” Maynooth University (Ireland). March 5. (i)
- O’Donnell, Daniel Paul. 2015. “‘All Together Now...’ Mobilising the (digital) Humanities in the Information Age.” Brigham Young University. February 2. (i)
- O’Donnell, Daniel Paul. 2015. “‘All Together Now...’ Mobilising the (digital) Humanities in the Information Age.” University of Balamand (Lebanon). January 16. (i)
- O’Donnell, Daniel Paul. 2015. “First thing we do, let's kill all the authors. On subverting scientific and scholarly authorship.” Force 2015. Oxford. January 13. (r)
- O’Donnell, Daniel Paul. 2014. “‘All Together Now...’ Mobilising the (digital) Humanities in the Information Age” Universität Basel, October 13. http:/\allowbreak{}/\allowbreak{}www.slideshare.net/\allowbreak{}caedmon/\allowbreak{}20141013-basel. (i)
- O'Donnell, Daniel Paul, Alex Gil. “Globalisation.” Around the world. A worldwide collaborative conference on privacy and surveillance in the digital age. 2014. Hosted by the Kule Institute for Advanced Study. May 21. http:/\allowbreak{}/\allowbreak{}aroundtheworld.ualberta.ca/\allowbreak{}2014/\allowbreak{}06/\allowbreak{}ulethbridge-and-columbia/\allowbreak{} (i)
- Rockwell, Geoffrey, Michael Sinatra, Susan Brown, Dean Irvine, Ray Siemens, Stefan Sinclair, Daniel Paul O'Donnell. 2014. “Problems to be solved in DigHum: The Large-Scale Digital Humanities. Opportunities and challenges of distributed large-scale data to the humanities.” GRAND Ottawa. May 14. (i)
- O'Donnell, Daniel Paul, et al. 2014. “The Lethbridge Journal Incubator: A new business model for Open Access journal publication.” Elsevier Labs (Virtual presentation). February 18. (i)
- Hobma\*, Heather and Daniel Paul O'Donnell. 2014. “Living out loud: The Visionary Cross Project and the Public Humanities.” CMRS/\allowbreak{}ETRUS. University of Saskatchewan. Saskatoon. January 16. (is)
- O'Donnell, Daniel Paul, et al. 2014. “Class 2.0: Digital technology & digital rhetorics in the undergraduate classroom.” Department of English, University of Saskatchewan. Saskatoon. January  15. Also presented to the Department of English, University of Lethbridge. February 7, 2014. (i)
- O'Donnell, Daniel Paul, et al. 2013. “The Lethbridge Journal Incubator. Leveraging Open Access Publication to Increase the Training and Research Capacity of the University.” Western Humanities Associate. University of California San Diego. November 1, 2013. (i)
- Leoni\*, Chiara, Marco Callieri, Matteo Dellepiane, Roberto Rosselli del Turco, Daniel Paul O’Donnell, and Roberto Scopigno. 2013. “The Dream and the Cross: A 3D-Referenced, Web-Based Digital Edition.” Digital Heritage International Conference 2013. Marseille. Monday, October 28. This also won best paper prize. (rs)
- O'Donnell, Daniel Paul. 2013. “This changes everything! The “Digital Turn” and the Institutional Practice of the Humanities.” I Seminário Internacional em Humanidades Digitais no Brasil. Universidade de São Paulo. October 23. (i)
- Daniel Paul O'Donnell, et al. 2013. “The The Visionary Cross Project.” ISAS 2013 (International Society of Anglo-Saxonists). Trinity College and University College Dublin.  29 July. (r)
- O'Donnell, Daniel Paul, et al. 2013. “The Unessay, or, The Pedagogy of Screwing Around. A Digital Humanities Approach to Teaching Scholarly Writing.” Pedagogy Lightening Talks. Digital Humanities 2013. University of Nebraska. 17 July.
- O'Donnell, Daniel Paul. 2013. “Everything that Rises Must Converge: On the Convergence of Informational and Critical Approaches to Textual, Cultural, and Material Heritage.” Medieval Cultural, Textual, and Material Culture in the Digital Age. Leeds International Medieval Congress. University of Leeds. 1 July, 2013. Different Version also delivered to Social Digital Scholarly Editing 2013. University of Saskatoon. 11 July. (r)
- O'Donnell, Daniel Paul. 2013. “The end(s) of history: A case study in the practice of digital popular knowledge mobilization.” Refereed Lecture. Immersive interpretation and the small cultural heritage site: the case of Ruthwell Kirk. Canadian Society for Digital Humanities. Victoria. 5 June. (r)
- O'Donnell, Daniel Paul. 2013. “The Old Familiar Faces: On the consumption of (digital) textual scholarship.” Refereed Lecture. Social and Digital Editions. Canadian Society for Digital Humanities. Victoria. 3 June. (r)
- O'Donnell, Daniel Paul. 2013. “The meteor has struck. The dust is in the air. Let’s leave the dinosaurs to their fate and concentrate on the mammals: Notes on the New Humanities.” Invited Keynote. Digging the Digital. Graduate Student Conference on the Digital Humanities. University of Alberta. April 5-6. (i)
- O'Donnell, Daniel Paul. 2013. “New Business Models for Open Access Publication in the Humanities: The Lethbridge Journal Incubator.” Beyond the PDF2. Amsterdam. 19 March. (i)
- O'Donnell, Daniel Paul, Catherine Karkov, and Heather Hobma\*. “Far From the Maddening Crowd: Digital Projects and the Ethics of Popular Knowledge Mobilization.” Havana. 12 December, 2012. (rs)
- O'Donnell, Daniel Paul, James Graham, Catherine E. Karkov, and Roberto Rosselli del Turco. 2012. "Is there a text in this edition? On the implications of multiple media and immersive technology for the future of the ‘scholarly edition’." European Society for Textual Scholarship (ESTS), Amsterdam. 23 November. (r)
- O'Donnell, Daniel Paul. 2012. "The Lethbridge Journal Incubator: Aligning scholarly publishing with the teaching and research missions of a public university." Canadian Association of Learned Journals. Congress of the Federation of the Social Sciences and Humanities. Waterloo, Ontario. May 27. (i)
- O'Donnell, Daniel Paul. 2012. "Markup and Metadata: An introduction to the power of XML and related technologies in humanities research applications." Pre-conference Workshop. Society for Digital Humanities/\allowbreak{}Société pour l’étude des médias interactifs. Congress of the Federation of the Social Sciences and Humanities. Waterloo, Ontario. May 25. (i)
- O'Donnell, Daniel Paul. 2012. "abdah.org: Alberta Digital Arts and Humanities and Campus Alberta." Presentation to Campus Alberta Arts, Social Sciences, and Humanities. University of Calgary. May 11. (i)
- O'Donnell, Daniel Paul. 2012. "Move Over: Learning to Read (and Write) with Novel Technology." MARCS: The Medieval and Renaissance Cultural Studies Research Group. University of  Calgary, March 15. (i)
- O'Donnell, Daniel Paul. 2011. "'Nor doubted once': Editing Text and Context." INKE Research Foundations For Understanding Books And Reading In A Digital Age Text And Beyond. Ritsumeikan University, Kyoto, Japan. November 18th. (r)
- O'Donnell, Daniel Paul. 2011. “The Medieval Academy's Digital Initiatives.” Invited Lecture. Medieval Electronic Scholarly Alliance (MESA). Mellon Workshop, Baltimore, May 2. (i)
- O'Donnell, Daniel Paul. 2011. “Digital Humanities and 'The Digital Humanities', Or, Should a Digital Humanities Center be Concerned with Word Processing.” Invited Lecture. Texas A&M University. February 17. (i)
- O'Donnell, Daniel Paul. 2010. “What Comes Between: Editing Context.” 7th Annual Conference. European Society for Textual Scholarship. Pisa, Italy. November. (r)
- O'Donnell, Daniel Paul. 2010. “What is a Markup Language, Really? A Generic and Modular Approach to Understanding Markup Semantics” TEI Annual Conference and Members' Meeting, Zadar, Croatia. November. (r)
- O'Donnell, Daniel Paul. 2010. “And everybody goes ahh: thinking outloud about interoperability.” Response paper. Digitized Collections of Medieval Manuscripts, A Workshop on Uses and Interoperation. Paris, January 14-15, 2010. http:/\allowbreak{}/\allowbreak{}lib.stanford.edu/\allowbreak{}DMSS (https:/\allowbreak{}/\allowbreak{}lib.stanford.edu/\allowbreak{}files/\allowbreak{}ODonnellMellon1Smaller.pdf) (i).
- O'Donnell, Daniel Paul. 2009. "TEI: What and Why?" Invited Lectures and Workshop. Textual heritage and modern information technologies. Izhevsk, Russia. October 11-15. (i)
- O'Donnell, Daniel Paul. 2009. "Exploiting TEI Markup." Invited Lectures and Workshop. Textual heritage and modern information technologies. Izhevsk, Russia. October 11-15. (i)
- O'Donnell, Daniel Paul. 2009. "Cædmon's Hymn: Project Management and Development." Invited Lectures and Workshop. Textual heritage and modern information technologies. Izhevsk, Russia. October 11-15. (i)
- O'Donnell, Daniel Paul. 2009. "Sugar and Spice and... Sausage filling. What the TEI is made of." Invited Lecture.     Early Chán Manuscripts among the Dūnhuáng Findings– Resources in the Mark-up and Digitalization of Historical Texts. Oslo, September 30. (i)
- O'Donnell, Daniel Paul. 2009. "Are you sure we're not in Kansas any more, Dorothy? Domain knowledge and the future of the Digital Humanities." (Invited Lecture and Workshop). University of South Carolina September 24. (i)
- O'Donnell, Daniel Paul. 2009. "Sugar and Spice and... Sausage filling. What the TEI is made of" (Institute Lecture). Invited Lecture. Digital Humanities Summer Institute, University of Victoria. June 6. (i)
- O'Donnell, Daniel Paul. 2009. "We are Family: Digital Medievalist as Community of Practice." Medieval Studies and New Media /\allowbreak{} Les Médiévistes et les Nouveaux Media. ENS-Lyon. March 31. (i)
- O'Donnell, Daniel Paul. 2009. "Mind the Gap: Editing the spaces between objects in a post print world" (Keynote). Beyond Analogue: Current Graduate Research in Humanities Computing. University of Alberta. February 13. (i)
- O'Donnell, Daniel Paul. 2009. "Mind the Gap: Representing the Relationships among Constituents in a Multi-Object Digital Edition" (Keynote). Incontri di Filologia digitale. Università degli Studi di Verona, January 15. (i)
- O'Donnell, Daniel Paul. 2008. "Standoff Markup." Invited participant. Roundtable. CASTA. University of Saskatchewan. October 18. (i)
- O'Donnell, Daniel Paul. 2007. "'Murder to dissect'?: Digitisation as a Theory of the Text." SDH/\allowbreak{}SÉMI 2007, University of Saskatchewan. May 29. (r)
- O'Donnell, Daniel Paul, Catherine Karkov, James Graham, Wendy Osborn, Roberto Rosselli Del Turco, Dot Porter. 2007. "The Visionary Cross: An Experiment in the Multimedia Edition." Digital Humanities 2007. University of Illinois, Urbana-Champaign. June 5. (r)
- O'Donnell, Daniel Paul. 2006. “We are family: the economics of best practice.” Invited Lecture. TEI Members Meeting, Victoria BC. October 27. (i)
- O'Donnell, Daniel Paul. 2006. “How Digital must a digital edition be?” 41st International Congress on Medieval Studies. University of Western Michigan. May 7. (r)
- O'Donnell, Daniel Paul. 2006. “Why should I write for your Wiki.” Renaissance Society of America. San Francisco. March. (i)
- O'Donnell, Daniel Paul. 2005. “Using Electronic Media to Improve Efficiency and Intelligibility in Teaching and Researching the Middle Ages” [Satirical lecture]. Societas Fontibus Historiæ medii Aevii Inveniendis, vulgo dicta “The Pseudo Society”. 40th International Congress on Medieval Studies, Western Michigan University (Kalamazoo), May 7.
- O'Donnell, Daniel Paul. 2004. “Back to the future: what electronic editors can learn from print editions of texts in multiple versions.” European Society for Textual Studies Conference. Alicante, Spain. November 24-25. (r)
- O'Donnell, Daniel Paul. 2004. “Best Practice in the Production of Digital Resources for Medievalists: Theory and Application”. Thirty-ninth International Congress on Medieval Studies (Kalamazoo). May. (r)
- O'Donnell, Daniel Paul. 2004. “Best Practice in the Production of Digital Resources for Medievalists: Project Design, Management, and Implementation”. Thirty-ninth International Congress on Medieval Studies (Kalamazoo). May. (r)
- O'Donnell, Daniel Paul. 2004. “Now What?: The Digital Medievalist Project and the Discovery of Best Practice.” Invited Lecture. SSHRC/\allowbreak{}University of Calgary ITST Summer Institute. May 26. (i)
- O'Donnell, Daniel Paul. 2004. “The Electronic Cædmon's Hymn: A Single Scholar, Multiple Text Electronic Edition and Archive.” Invited Lecture. SSHRC/\allowbreak{}University of Saskatchewan ITST Conference, May 15. (i)
- O'Donnell, Daniel Paul. 2004. “The Text Encoding Initiative: A Theoretical Standard for the Encoding of Electronic Texts.” Invited Lecture. SSHRC/\allowbreak{}University of Saskatchewan ITST Conference, May 15. (i)
- O'Donnell, Daniel Paul. 2004. “Poetry, Prose, and Book History: A way forward in debates about scribal literacy in Anglo-Saxon England?” Poetry and Prose: Intersections (to 1100): Methods and Approaches (Organisers: Carin Ruff and Elizabeth M. Tyler). Thirty-ninth International Congress on Medieval Studies (Kalamazoo). May 8. (r)
- O'Donnell, Daniel Paul. 2003. “Texts and the Single Scholar: Is the morning after worth the night before?” [Lecture on electronic project management]. Thirty-eighth International Congress on Medieval Studies (Kalamazoo). May 8. (r)
- O'Donnell, Daniel Paul. 2002. “'Vade retro me Satana': What the Norton Anthology of Poetry gets Wrong on Page 1.” Department of English Colloquium series. University of Lethbridge. November 28.
- O'Donnell, Daniel Paul. 2002. “The Cædmon Code: Some Problems with Numerical and Geometrical Patterning in an Early Medieval Text.” Working Papers in the Humanities Colloquium. University of Lethbridge. April 4.
- O'Donnell, Daniel Paul. 2001. “‘Is that your final answer?’ Reconstructing Cædmon’s Hymn in a Post-Modern Age.” Germanic Philology Session, MLA Annual Meeting. Washington D.C., December 27, 2000. Also delivered the Department of English Research Colloquium, January.
- O'Donnell, Daniel Paul. 2000. “Text and Context: Generic Factors affecting Scribal Performance in the Transmission of Old English Verse.” Department of English Research Colloquium. University of Lethbridge. November 15.
- O'Donnell, Daniel Paul. 2000. “The Editor Function, or I Know More about Cædmon’s Hymn than you do, Nyah, Nyah!” Inaugural Lecture, Humanities Computing Series, University of Calgary, June 15. (i)
- O'Donnell, Daniel Paul. 2000. “Reading Bede Reading Cædmon: Understanding a Critical Miracle.” Thirty-fifth International Congress on Medieval Studies (Kalamazoo). May 8. This was a thoroughly revised an abridged version of my March 1999 Departmental lecture and April 2000 lecture at Queen’s. (r)
- O'Donnell, Daniel Paul. 2000. “Reading Bede Reading Cædmon: Bede’s Historia ecclesiastica as Source and Source of Interpretation for Cædmon’s Hymn.” Invited Lecture. Department of English, Queen’s University. April 10. (i)
- O'Donnell, Daniel Paul. 2000. “The Editor Function: Form, Content, and Editorial Theory in Editing Cædmon’s Hymn.” Department of English Research Colloquium. University of Lethbridge. February 2. This was a revised version of my May 9, 1999 invited lecture at the Kalamazoo Medieval Studies Conference.
- O'Donnell, Daniel Paul. 1999. “The Editor Function: Form, Content, and Editorial Theory in Editing Cædmon’s Hymn.” Commissioned Lecture. Thirty-fourth International Congress on Medieval Studies (Kalamazoo). May 9. (r)
- O'Donnell, Daniel Paul. 1999. “Reading Bede Reading Cædmon: Bede’s Historia ecclesiastica as Source and Source of Interpretation for Cædmon’s Hymn.” Department of English Research Colloquium. University of Lethbridge. March.  A lightly revised version of this paper was delivered as an invited lecture, April 10, 2000 at Queen’s University.
- O'Donnell, Daniel Paul. 1998. “Fish and Fowl: Generic Expectations and the Relationship between the Old English Phoenix poem and Lactantius’s de ave phoenice.” Germania Latina IV. Groningen, The Netherlands, July. (i)
- O'Donnell, Daniel Paul. 1998. “What Anne Meant: Generic Instability and the Transmission of Anne Frank’s Diary.” Department of English Research Colloquium. University of Lethbridge. March.
- O'Donnell, Daniel Paul. 1997. “A New Theory of Poetic Textual Transmission.” Delivered at: “Anglo-Saxon Studies in the Twentieth Century.” International Society of Anglo-Saxonists Conference (Palermo, Italy). July 11. (r)
- O'Donnell, Daniel Paul. 1996. “The Text of Cædmon’s Hymns.” Delivered at: “Focusing on Editorial Scholarship at the Century’s End.” MLA Convention, Washington D.C. December 28. (r)
- O'Donnell, Daniel Paul. 1996. “‘Transitional Literacy’ and the Poems of the Anglo-Saxon Chronicle: Context as Counter-evidence.” Medieval Chronicle Conference. Rijksuniversiteit Utrecht/\allowbreak{}Driebergen. July 13. (r)
- O'Donnell, Daniel Paul. 1996. “Ends and Means: Manuscript Context and Scribal Accuracy in the Copying of Anglo-Saxon Poetry.” Invited Lecture. Department of English, Trinity College Dublin. March 19. (i)
- O'Donnell\*, Daniel Paul. 1994. “The Spirit and the Letter: The Use of the Dramatic in Old Frisian Legal Writing.” Twenty-ninth International Congress on Medieval Studies (Kalamazoo). May 4. (i)
- O'Donnell\*, Daniel Paul. 1991. “Beinecke MS 594: A Second Look at a Well-Known Nominale.” Early Book Society Conference. Trinity College Dublin. (r)
- O'Donnell\*, Daniel Paul. 1991. “The OE ‘Phoenix’, Lactantius’s ‘De ave phoenice’ and the Science of Allegory,” Harvard-Yale Graduate Student Colloquium.

\sectionbreak{}
## Other Lectures (Partial List)

- O’Donnell, Daniel Paul. 2022. “Solidarity: Some thoughts on the recent strike and lockout at the U of Lethbridge.” Northern Alberta Institute of Technology. April 21. (i)
- O’Donnell, Daniel Paul. 2021. “You can’t handle the truth! Academic freedom and University Governance.” Mount Royal University Faculty Association. March 5. (i)
- O’Donnell, Daniel Paul, Paul Hayes, Andrea Amelinckx, Annabree Fairweather, Terry Sway, Rumi Graham. 2018. “I am he as you are he and you are me and we are all together: Communicating with professional know-it-alls.” CAFA Labour Conference. March 9. (i)
- O'Donnell, Daniel Paul. 2018. “Know your song well before you start singin’: Negotiating when ‘black is the color’ and ‘none is the number.’” CAUT Bargaining Roundtable. (i)
- O'Donnell, Daniel Paul, Paul Hayes. 2017. “Understanding Lockout/\allowbreak{}Strike in the Post-Secondary Sector under the Alberta Labour Relations Code.” ULFA. Lethbridge, Canada. https:/\allowbreak{}/\allowbreak{}zenodo.org/\allowbreak{}record/\allowbreak{}1045544. (i)

\sectionbreak{}
## General Interest and Guest Undergraduate Lectures (Partial List)

- “The cipher manuscript”: Hoaxes as a book history problem.” Invited guest lecture. English 4400. Book History. Nov. 6, 2021. https:/\allowbreak{}/\allowbreak{}doi.org/\allowbreak{}10.5281/\allowbreak{}zenodo.5651109. (i)
- “Beowulf: Manuscripts, Pronunciation, Grammar, and Metre.” Invited guest lecture. English 1900. University of Lethbridge. Sept. 17, 2014. (i)
- “The Importance of Being Earnest: Coding Problems and Solutions.” Invited Guest Lecture. English 517/\allowbreak{}607 (Graduate Humanities Computing), University of Calgary. June 15, 2000. (i)
- “Tolkien’s Elvish and Other Constructed Languages.” Invited Lecture. English 3700 (Children’s Literature), University of Lethbridge. February 7, 2000. (i)
- “Anne, Otto, and the Neo-Nazis. The (Im)morality of Reading a Holocaust Diary.” Invited Lecture. Arts and Science 1000 (Liberal Arts), University of Lethbridge. November 2001; September 2000; November 1999. (i)
- “What Anne Meant: Generic Instability and the Transmission of Anne Frank’s Diary.” Invited Lecture. Arts and Science 1000 (Liberal Arts), University of Lethbridge. November 2001; September 2000; November 1999. (i)
- “Beowulf and its Place in History.” Worker’s Educational Association Weekend School at Horncastle. February 1997. (i)

\sectionbreak{}
## Posters (Partial List)

- “Hier stehe ich! Operationalising conviction in the Scholarly Commons.” Michael Bar-Sinai, Jeroen Bosman, Ian Bruno, Chris Chapman, Bastian Greshake Tzovaras, Stephanie Hagestrom, Nate Jacobs, Bianca Kramer, Maryann Martone, Fiona Murphy, Daniel Paul O’Donnell. El Pub 2018 Toronto. http:/\allowbreak{}/\allowbreak{}doi.org/\allowbreak{}10.5281/\allowbreak{}zenodo.1292850. (r)
- “The DH Experience.” Selman Zachary Palmer\*, Geoffrey Rockwell, Stan Ruecker, John Montague, Luciano Frizzera, Alex Gill, Daniel O’Donnell. DH 2015. Melbourne. July 2015. (r)
- “The Lethbridge Journal Incubator.” DH 2013. Lincoln, Nebraska. July 2013. (i)
- “Embedding Expert Knowledge in a TEI Apparatus.” TEI Members Meeting and Conference. University of Zadar, Croatia. November 2010. (r)
- “The Digital Medievalist Project: An Experiment in Digital Community.” Digital Philology and Medieval Texts. Arizzo. January 19–21, 2006. (r)
- “Digital Medievalist: The Scholarly Journal as Community of Practice.” MLA Convention. Washington, D.C. December 27–29, 2005. (r)
- “TEI/\allowbreak{}XML as a Production Language: The Case of the Digital Medievalist Project.” TEI Annual Meeting. Sofia, Bulgaria. October 28–29, 2005. (r)
- “The Digital Medievalist Project: An Extensible Community of Practice for Medievalists.” Poster and abstract. Digital Resources in the Humanities Conference. University of Newcastle upon Tyne, UK. September 5–9, 2004. [3rd Prize winner in best poster competition (out of 22 entries)]. (r)

\sectionbreak{}
# Teaching and Supervisory Experience

**Summary**: Daniel Paul O’Donnell has taught undergraduate and graduate courses in English literature, medieval studies, English linguistics, textual scholarship, management information studies, mathematics, computer science, and the digital humanities. His teaching appointments have included Yale University, the Workers’ Educational Association (U.K.), the University of York (U.K.), Louisiana State University, and the University of Lethbridge, as well as invited roles with the International Society for Anglo-Saxonists. He has also supervised more than thirty M.A. and M.Sc. students in editorial, adjudication, and publication work through the *Meeting of Minds* Graduate Student Journal and his leadership of the University of Lethbridge Journal Incubator. He served as lead instructor for all courses unless otherwise indicated.

\sectionbreak{}
## Supervision

\sectionbreak{}
### Postdoctoral Supervision

- Nathan Woods. Humanities data. 2021–

\sectionbreak{}
### PhD Supervision

- AKM Iftekhar Khalid. Tagorian English vs Large Language Models as Models of Hegemonic Discourse. CPST. Co-supervisor. 2024-
- Frank Onuh. Discourse analysis and hate speech. CPST. Co-supervisor. 2023–
- Davide Pafumi. Discourse analysis and Medieval English. CPST. Co-supervisor. 2022–

\sectionbreak{}
### M.A. Supervision

- Morgan Pearce. Chaucer. English. Co-supervisor. 2022–
- Kirandeep Kaur. Language resources for lesser-supported languages. English. Supervisor. 2021–2023
- AKM Iftekhar Khalid. The History of English in Bangladesh. English. Supervisor. 2021–2023
- Virgil Grandfield. Creative Non-fiction. Multidisciplinary. Supervisor. 2015–2018 (medical withdrawal)
- Gurpreet Singh. Crowdsourcing editions of Sikh religious texts. Multidisciplinary. Supervisor. 2014–2017 (medical withdrawal)
- Rylan Spenrath. The Monstrous in Fiction. English. Supervisor. 2013–2016
- Heather Hobma. Ruthwell Kirk. English. Supervisor. 2011–2014
- Tania Bigthroat. Residential School Experience. Native American Studies. Co-supervisor. 2005–2008
- Shelley Stigter. Native American Verbal Art. English. Supervisor. 2004–2006

### Committee Membership (M.A. and M.Sc.)

- Mahsa Miri. Crediting Systems in Global Film. Modern Languages/\allowbreak{}Film Studies. 2015–2019
- Titi Babalola. Absurd Realism. English. 2012–2014
- Jessica Bay. Fan Fiction. English. 2012–2014
- Fatima Rahman. Region query processing. Computer Science. 2013– (with interruptions)
- Cody Rioux. Text Extraction in Digital Libraries. Computer Science. 2013–2015
- Kent Aardse. Contemporary Digital Literatures. English. 2009–2011
- Drew Luby. Magic in Renaissance England. English. 2009–2011
- Leanne Little. Masculinity in Shakespeare. English. 2007–2009
- Rob Meckelborg. Satanic Blake. English. 2006–2007
- Angela Mlynarski. Text Summation in Digital Libraries. Computer Science. 2004–2006 (direction assumed during writing stage)
- Laura Cappello. Jane Austen. English. 2002–2004


\sectionbreak{}
## University of Lethbridge (1997–)

- *Meeting of Minds* Graduate Student Journal (2015–2016). Supervised and instructed graduate students in contemporary publication techniques. The work produced two student-co-authored articles.

- English 1900. Introduction to English Language and Literature. (Undergraduate)
- English 2100. Poetry. (1997–2007) (Undergraduate)
- English 2450. English Literature Survey I (until 1800). (2011–) (Undergraduate)
- English 2810. Grammar. (2000–) (Undergraduate)
- English 2900. World Englishes. (1997–2005) (Undergraduate)
- English 3401. Middle English. (1997–) (Undergraduate)
- English 3450. Old English. (1997–) (Undergraduate)
- English 3601. Chaucer. (1997–) (Undergraduate)
- English 3901. History of English. (1997–) (Undergraduate)
- English 3990. Independent Studies. (1997–) (Undergraduate)
- English 4400/\allowbreak{}4600. Beowulf. (1997–) (Undergraduate)
- English 4400. History of the Book. (1998–) (Undergraduate)
- English 4400. Digital Humanities. (2005–) (Undergraduate)
- English 4400. Scholarly Communication. (2017–) (Undergraduate)
- English 4900. Independent Studies. (1997–) (Undergraduate)
- Management 4900. Database Design. (1998) (Undergraduate)
- English 5400. The New Humanities. (2013) (Graduate)
- English 5400. Digital Humanities. (2005–) (Undergraduate)
- English 5400. Scholarly Communication. (2017–) (Undergraduate)
- English 5600. Beowulf. (2014) (Graduate)
- English 5990. Graduate Independent Studies. (2002–) (Graduate)

\sectionbreak{}
## University of York (U.K.) (1996)

- Middle English paper (Undergraduate)

\sectionbreak{}
## Louisiana State University (1994–1995)

- Old English (Graduate)
- Beowulf (Graduate)
- Chaucer (Undergraduate)
- History of English (Undergraduate)

\sectionbreak{}
## Yale University (1991–1992)

- Introduction to the Novel (Teaching Assistant) (Undergraduate)
- Age of Johnson (Teaching Assistant) (Undergraduate)
- History of English (Teaching Assistant) (Undergraduate)

\sectionbreak{}
## Other

- Workers’ Educational Association. Dark Age Tales (1996) (General Interest)
- Co-director, with Martin Foys (Drew University). ISAS Pre-conference PhD Workshop in Digital Technology and New Media. July–August 2009.

\sectionbreak{}
# Consulting

- Readex/\allowbreak{}Newsbank. Interactive Web. 2006-2009.
- Citec. SGML. 1999.
- Kawoosh! Productions/\allowbreak{}Stargate SG-1 [Television Producers]. Historical language use; pronunciation; archaeology. 1999.

\end{document}
