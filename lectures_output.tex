\cvitem{[142]}{ i Daniel Paul O’Donnell. 2024. “Academic Freedom, Privilege, and Intersectionality.” July 29.  Force11 Scholarly Communications Institute (FSCI), Los Angeles. Zenodo. https://doi.org/10.5281/zenodo.13121345}

\cvitem{[141]}{ rs Bordalejo, Barbara, Davide Pafumi, Frank Onuh, AKM Iftekhar Khalid, and Daniel O’Donnell. 2024. “Scarlet Cloak and the Forest Adventure: The Issue of False Positives in AI Detection Tools.” June 17. Congress of the Humanities and Social Sciences (CSDH/SCHN). Montreal. https://doi.org/10.5281/zenodo.12011536.}

\cvitem{[140]}{ rs Bordalejo, Barbara, Davide Pafumi, Morgan Slayde Pearce, and Daniel O’Donnell. 2024. “Adapting a Research Tool for Teaching in a Post-Pandemic World.” June 17. Congress of the Humanities and Social Sciences (CSDH/SCHN). Montreal.  https://doi.org/10.5281/zenodo.12018101.}

\cvitem{[139]}{ R O’Donnell, Daniel Paul, Barbara Bordalejo, and Nathan D. Woods. 2024. “Data in/And Mediaeval Studies.” International Congress on Medieval Studies. Kalamazoo. May 11. https://doi.org/10.5281/zenodo.11179271.}

\cvitem{[138]}{ R Woods, Nathan, Barbara Bordalejo, and Daniel O’Donnell. 2023. “Data Problems in the Humanities, or ‘When Everybody Is Special, No One Is’?” October 3. https://doi.org/10.5281/zenodo.8403789.}

\cvitem{[137]}{ r Bordalejo, Barbara, Daniel Paul O’Donnell, and Nathan Woods. 2023. “The Implications of Multiple Hierarchies for the Future of Humanities Data: Or, What Is a Markup Language, Actually?” October 5. https://doi.org/10.5281/zenodo.8411165.}

\cvitem{[136]}{ r O’donnell, Bordalejo, and Woods. 2023. “Representational Data:  A Case Study.” May 30.  Congress of the Social Sciences and Humanities. https://doi.org/10.5281/zenodo.7986430.}

\cvitem{[135]}{ r	Daniel Paul O’Donnell, Barbara Bordalejo, Nathan Woods, Roberto Rosselli Del Turco. 2022. “Small Data projects/Big Data research: contemporary problems and historical solutions.” DH 2022 (Tokyo/online). July 28. https://doi.org/10.5281/zenodo.6857202.}

\cvitem{[134]}{ Daniel Paul O’Donnell. 2022. “‘Good things come in small packets’: Investigating Humanities Research Data Practices: A Community of Practice Approach.” FSCI 2022. Lightning Talks. July 25-29. https://doi.org/10.5281/zenodo.6902821.}

\cvitem{[133]}{ i	Daniel Paul O’Donnell. 2022. “REPO: Reimagining Educational Practices for Open 2020-2021.” FSCI 2022. Force11 Working Group Bazaar. July 25. https://doi.org/10.5281/zenodo.6902861.}

\cvitem{[132]}{ r	Daniel Paul O’Donnell, Barbara Bordalejo, Nathan Woods, Roberto Rosselli Del Turco. 2022. “Small Data Management in Humanities and Cultural Heritage Projects.” DH Unbound. May 17. https://doi.org/10.5281/zenodo.6773410.}

\cvitem{[131]}{ i 	Daniel Paul O’Donnell. 2022. “Thinking about CARE principles in the Digital Humanities.  Why CARE may not be only a matter for researchers working with indigenous peoples.” RDA-DU. Online. Feb 21.}

\cvitem{[130]}{ i	Daniel P. O'Donnell. 2021. ''Thinking about CARE principles in the Digital Humanities. Why CARE may not be only a matter for researchers working with indigenous peoples.” DARIAH Friday Frontiers. Online. October. France. October 8.}

\cvitem{[129]}{ i	Daniel P. O'Donnell, Rosselli Del Turco R. 2020. 'Methods and tools to build a web-based edition on the basis of FAIR/OA data'. Assemblée Générale du consortium Cahier. France. November 27. https://cahier.hypotheses.org/5384.}

\cvitem{[128]}{ i	O'Donnell, Daniel. (2020, August). How to make a manifesto… and not: Principles of the Scholarly Commons as Case History (Version Presentation). Presented at the Force11 Scholarly Communication Institute 2020 Online (FSCI 2020 Online), Online: Zenodo. http://doi.org/10.5281/zenodo.3979532.}

\cvitem{[127]}{ r Bordalejo, Barbara, Folgert Karsdorp, Daniel ODonnell, Basten Stokhuyzen, and Karina Van Dalen-Oskam. 2020. “Check Your Privilege: The Digital Privilege Game.” Congress 2020 (online). June 5. Published in Building Community Online. https://hcommons.org/deposits/item/hc:30179.}

\cvitem{[126]}{ i O’Donnell, Daniel, and Virgil Grandfield. 2020. “The Lethbridge Journal Incubator: A Collaborative Student-Expert Open Access Publishing Model.” Open Publishing Fest. May 28. https://doi.org/10.5281/zenodo.3863022.}

\cvitem{[125]}{ i 	O’Donnell, Daniel Paul. 2019. Publishing (and Forgetting) the Small or Medium-sized Scholarly Edition or Cultural Heritage Collection as Linked Open Data: Using Zenodo and Github to Publish the Visionary Cross Project. Winter Institute in Digital Humanities. IIT Gandhinagar. Palaj, India. December 20. https://zenodo.org/record/3586007. (Updated version of DH2019 presentation).}

\cvitem{[124]}{ i 	O’Donnell, Daniel Paul. 2019. Small, thick, and slow: Towards an Open and FAIR data culture in the Humanities. Winter Institute in Digital Humanities. IIT Gandhinagar. Palaj, India. December 18. https://zenodo.org/record/3581520.}

\cvitem{[123]}{ r 	O’Donnell, Daniel Paul. 2019. Publishing (and Forgetting) the Small or Medium-sized Scholarly Edition or Cultural Heritage Collection as Linked Open Data: Using Zenodo and Github to Publish the Visionary Cross Project. Digital Humanities 2019. Utrecht, Netherlands. July 10. https://zenodo.org/record/3586007.}

\cvitem{[122]}{ I	O’Donnell, Daniel Paul. 2019. Small, thick, and slow: Thinking about data and research publication in the Humanities in the age of Open and FAIR. Curtin University. Perth, Australia. November 25, 2019. https://zenodo.org/record/3554760.}

\cvitem{[121]}{ i 	O’Donnell, Daniel Paul. 2018. The Nature of Humanities Data. Open Science Infrastructures for Big Cultural Data: International Advanced Masterclass. Plovdiv, Bulgaria. December 14. http://doi.org/10.5281/zenodo.2246390.}

\cvitem{[120]}{ i 	O’Donnell, Daniel Paul. 2018. The Humanities and New Technologies: Exploring Digital Tools for Innovation and Development. Keynote address. 2nd Lagos Summer School in Digital Humanities. University of Lagos, Nigeria. October 2. https://doi.org/10.5281/zenodo.1443255.}

\cvitem{[119]}{ i	O’Donnell, Daniel Paul. 2018. Open Access/Open Science and the Humanities on a Cross-Regional Basis. Invited lecture. Elizade University, Ilara-Mokin, Nigeria. Sept. 28. https://doi.org/10.5281/zenodo.1455735.}

\cvitem{[118]}{ i	O’Donnell, Daniel Paul. 2018. Sometimes a poet is just a poet: Reading Bede reading Cædmon’s inspiration in light of Germanic Tradition and cultural analogues. Invited lecture. XIX Seminario Avanzato in Filologia Germanica "Sogni, Visioni e Profezie nella letteratura Germanica medievale. Università degli studi di Torino. Sept. 18. http://doi.org/10.5281/zenodo.2222396.}

\cvitem{[117]}{ r	O’Donnell, Daniel Paul. 2018. Open in the Digital Humanities. Some thoughts on cross-disciplinary similarity and difference. DH 2018. Mexico City.}

\cvitem{[116]}{ i	O’Donnell, Daniel Paul. 2018. Le goût de la mer: Teaching Research. Invited lecture. University of Saskatchewan. May 8.}

\cvitem{[115]}{ isc Bar-Sinai, Michael, Jeroen Bosman, Ian Bruno, Chris Chapman, Bastian Greshake Tzovaras, Stephanie Hagstrom, Nate Jacobs, Bianca Kramer, Maryann Martone, Fiona Murphy, Daniel Paul O'Donnell [alphabetical order]. 2018. “Hier stehe ich! Operationalising conviction in the Scholarly Commons.” McGill. April 4.}

\cvitem{[114]}{ i	O'Donnell, Daniel Paul 2017. 'How to Start an Open Access Journal'. Open Access Week. Lethbridge. Canada.}

\cvitem{[113]}{ rs Heather Bliss, Inge Genee, Marie Odile Junker & Daniel Paul O’Donnell, 2017. “Credit where credit is due”: How to handle attribution, copyright, and intellectual property in on-line digital resources for Algonquian languages.” The Algonguin Conference, Université de Montréal. October 28.}

\cvitem{[112]}{ i 	O’Donnell, Daniel Paul. 2017. Humanities in the Age of Technology [Keynote]. 1st Lagos Summer School in Digital Humanities. University of Lagos, Nigeria. July 11.}

\cvitem{[111]}{ rs Esau, Paul, Carey Viejou, Sylvia Chow, Jarret McKinnon, Reed Parsons, and Daniel Paul O'Donnell. 2017. First World Problems: Publishing a Graduate Student Journal in the Anglophone Global North Using Minimal Computing Techniques. Session 9a: Open Access/Social Scholarship. Canadian Society for Digital Humanities/Société canadienne des humanités numériques. Congress 2017. Ryerson University, Toronto. May 31.}

\cvitem{[110]}{ rsc O’Donnell. Daniel Paul, Dot Porter, Roberto Rosselli Del Turco, Gurpreet Singh. 2017. Let’s Get Nekkid! Stripping the User Experience to the Bare Essentials. Sesion 5a: Digital Editions. Canadian Society for Digital Humanities/Société canadienne des humanités numériques. Congress 2017. Ryerson University, Toronto. May 30.}

\cvitem{[109]}{ rs Esau, Paul, Carey Viejou, and Daniel Paul O’Donnell. 2017. Publishing the Unpublishable: The Story of a Graduate Journal and a University that Wouldn’t Submit to it. Canadian Association of Learned Journals. Congress 2017. Ryerson University, Toronto. May 27.}

\cvitem{[108]}{ i O’Donnell, Daniel Paul. 2017. “The Bird in Hand: Humanities research in the age of open data.” Research Data Management Interdisciplinary Panel Discussion. Centre for the Study of Scholarly Communication. University Library. University of Lethbridge. March 16.}

\cvitem{[107]}{ i O’Donnell, Daniel Paul. 2017. “When is infrastructure really a project?.” Open Scientific Infrastructure: in Search of a Development Model. Gaidar Forum: Russia and the World: The Choice of Priorities. Russian Presidential Academy of National Economy and Public Administration. Moscow. January 14.}

\cvitem{[106]}{ i O’Donnell, Daniel Paul. 2016. “Length and Breadth: Why diversity is a core intellectual value in the Digital Humanities.” Intersectionality in Digital Humanities Conference. KU Leuven. September 15.}

\cvitem{[105]}{ r O’Donnell, Daniel Paul. 2016. All along the Watchtower: An interdisciplinary approach to understanding the importance technical, disciplinary, and interpersonal diversity within the Digital Humanities. DH 2016. Krakow. July 12.}

\cvitem{[104]}{ rs O’Donnell, Daniel Paul and Shelaigh Brantford. 2016. “The tip of the iceberg: Transparency and diversity in contemporary DH.” Congress 2016. University of Calgary. June 1. Summarised by Geoffrey Rockwell at http://philosophi.ca/pmwiki.php/Main/CSDH-CGSA2016.}

\cvitem{[103]}{ Bosman, Jeroen, Ian Bruno, Amy Buckland, Sarah Callaghan, Robin Champieux, Chris Chapman, Stephanie Hagstrom, Bianca Kramer, MaryAnn Martone, Daniel Paul O'Donnell. 2016. Scholarly Commons Working Group Webinar. Force11.org. May 24. https://www.force11.org/group/SCWG/May24webinar}

\cvitem{[102]}{ i O’Donnell, Daniel Paul. 2016. “All Growed Up: Preparing Gold Open Access Journals for a Post-Incubator Life. Some lessons on transitions and new beginnings.” OA@UNT. University of North Texas. May 20.}

\cvitem{[101]}{ i O’Donnell. Daniel Paul. 2016. The Visionary Cross Project and the Lethbridge Journal Incubator. Lethbridge SSHRC Showcase. April 29, 2016.}

\cvitem{[100]}{ ir O’Donnell, Daniel Paul. 2016. Est ce qu’il y a de hors-edition? or, Can you edit everything? MLA 2016 Austin. Session 215 Editing Unruly Objects. January 8.}

\cvitem{[99]}{ r Moore, Samuel, Damian Pattison, Cameron Naylon, Daniel O'Donnell. 2015. The Quality of Qualities. Excellence and Soundness in Scholarly Communication. Mellon Triangle Scholarly Communication Workshop. Durham/Raleigh. UNC. October 12-16.}

\cvitem{[98]}{ r Ortega, Élika, Alex Gil, Daniel Paul O'Donnell. 2015. “Psst! An Informal Approach to Expanding the Linguistic Range of the Digital Humanities.” Digital Humanities 2015. Melbourne. July 1.}

\cvitem{[97]}{ r O'Donnell, Daniel Paul, Gurpreet Singh, Rachel Hanks, Roberto Rosselli Del Turco. 2015. “The Old Familiar Faces: On the Consumption of Digital Scholarship.” Digital Humanities 2015. Melbourne. July 2.}

\cvitem{[96]}{ r O'Donnell, Daniel Paul. 2015. “Est ce qu’il y a de hors-edition? or, Can you edit everything?” Canadian Society for Digital Humanities. Congress of the Humanities and Social Sciences. Ottawa. June 3.}

\cvitem{[95]}{ O'Donnell, Daniel Paul, Rachel Hanks, Gurpreet Singh, Roberto Rosselli Del Turco. 2015. “The Old Familiar Faces: On the consumption of (digital) textual scholarship.” Highway 2 Conference, University of Lethbridge May 9.}

\cvitem{[94]}{ r O’Donnell, Daniel Paul. 2015. “The 'Unessay': A new approach to not teaching composition.” Spark. University of Lethbridge. April 30.}

\cvitem{[93]}{ i O’Donnell, Daniel Paul. 2015. ““If 'if's and 'and's were pots and pans...”: Aligning Open Access Publication with the Research and Teaching Missions of the Public University: The Case of the Lethbridge Journal Incubator.” Advancing Research Communication and Scholarship. Philadelphia. April 15.}

\cvitem{[92]}{ i O’Donnell, Daniel Paul. 2015. “‘Living Out Loud: Public Humanities and an Eight Century Cross.” North Carolina State University. April 15.}

\cvitem{[91]}{ i O’Donnell, Daniel Paul. 2015. “Critical Mass: Online Communities and the Advancement of Research Communication.” Maynooth University (Ireland). March 5.}

\cvitem{[90]}{ i O’Donnell, Daniel Paul. 2015. “‘Living Out Loud: The Visionary Cross Project and the Public Humanities.” Maynooth University (Ireland). March 5.}

\cvitem{[89]}{ i O’Donnell, Daniel Paul. 2015. “‘All Together Now...’ Mobilising the (digital) Humanities in the Information Age.” Brigham Young University. February 2.}

\cvitem{[88]}{ i O’Donnell, Daniel Paul. 2015. “‘All Together Now...’ Mobilising the (digital) Humanities in the Information Age.” University of Balamand (Lebanon). January 16.}

\cvitem{[87]}{ r O’Donnell, Daniel Paul. 2015. “First thing we do, let's kill all the authors. On subverting scientific and scholarly authorship.” Force 2015. Oxford. January 13.}

\cvitem{[86]}{ i O’Donnell, Daniel Paul. 2014. “‘All Together Now...’ Mobilising the (digital) Humanities in the Information Age” Universität Basel, October 13. http://www.slideshare.net/caedmon/20141013-basel.}

\cvitem{[85]}{ i O'Donnell, Daniel Paul, Alex Gil. “Globalisation.” Around the world. A worldwide collaborative conference on privacy and surveillance in the digital age. Hosted by the Kule Institute for Advanced Study. May 21, 2014. http://aroundtheworld.ualberta.ca/2014/06/ulethbridge-and-columbia/}

\cvitem{[84]}{ i Rockwell, Geoffrey, Michael Sinatra, Susan Brown, Dean Irvine, Ray Siemens, Stefan Sinclair, Daniel Paul O'Donnell. “Problems to be solved in DigHum: The Large-Scale Digital Humanities. Opportunities and challenges of distributed large-scale data to the humanities.” GRAND Ottawa. May 14, 2014.}

\cvitem{[83]}{ i O'Donnell, Daniel Paul, et al. “The Lethbridge Journal Incubator: A new business model for Open Access journal publication.” Elsevier Labs (Virtual presentation). February 18, 2014.}

\cvitem{[82]}{ i Hobma, Heather and Daniel Paul O'Donnell. “Living out loud: The Visionary Cross Project and the Public Humanities.” CMRS/ETRUS. University of Saskatchewan. Saskatoon. January 16, 2014.}

\cvitem{[81]}{ i O'Donnell, Daniel Paul, et al. “Class 2.0: Digital technology & digital rhetorics in the undergraduate classroom.” Department of English, University of Saskatchewan. Saskatoon. January  15, 2014. Also presented to the Department of English, University of Lethbridge. February 7, 2014.}

\cvitem{[80]}{ i O'Donnell, Daniel Paul, et al. “The Lethbridge Journal Incubator. Leveraging Open Access Publication to Increase the Training and Research Capacity of the University.” Western Humanities Associate 2013. University of California San Diego. November 1, 2013.}

\cvitem{[79]}{ r Leoni, Chiara, Marco Callieri, Matteo Dellepiane, Roberto Rosselli del Turco, Daniel Paul O’Donnell, and Roberto Scopigno. Forthcoming. “The Dream and the Cross: A 3D-Referenced, Web-Based Digital Edition.” Digital Heritage International Conference 2013. Marseille. Monday, October 28, 2013. This also won best paper prize.}

\cvitem{[78]}{ i “This changes everything! The “Digital Turn” and the Institutional Practice of the Humanities.” I Seminário Internacional em Humanidades Digitais no Brasil. Universidade de São Paulo.  October 23, 2013.}

\cvitem{[77]}{ r Daniel Paul O'Donnell, et al. “The The Visionary Cross Project.” ISAS 2013 (International Society of Anglo-Saxonists). Trinity College and University College Dublin.  29 July, 2013.}

\cvitem{[76]}{ O'Donnell, Daniel Paul, et al. “The Unessay, or, The Pedagogy of Screwing Around. A Digital Humanities Approach to Teaching Scholarly Writing.” Pedagogy Lightening Talks. Digital Humanities 2013. University of Nebraska. 17 July, 2013.}

\cvitem{[75]}{ r O'Donnell, Daniel Paul. “Everything that Rises Must Converge: On the Convergence of Informational and Critical Approaches to Textual, Cultural, and Material Heritage.” Medieval Cultural, Textual, and Material Culture in the Digital Age. Leeds International Medieval Congress. University of Leeds. 1 July, 2013. Different Version also delivered to Social Digital Scholarly Editing 2013. University of Saskatoon. 11 July, 2013.}

\cvitem{[74]}{s r O'Donnell, Daniel Paul. “The end of history: A case study in the practice of digital popular knowledge mobilization.” Refereed Lecture. Immersive interpretation and the small cultural heritage site: the case of Ruthwell Kirk. Canadian Society for Digital Humanities. Victoria. 5 June, 2013.}

\cvitem{[73]}{ r O'Donnell, Daniel Paul. “The Old Familiar Faces: On the consumption of (digital) textual scholarship.” Refereed Lecture. Social and Digital Editions. Canadian Society for Digital Humanities. Victoria. 3 June, 2013.}

\cvitem{[72]}{ i O'Donnell, Daniel Paul. “The meteor has struck. The dust is in the air. Let’s leave the dinosaurs to their fate and concentrate on the mammals: Notes on the New Humanities.” Invited Keynote. Digging the Digital. Graduate Student Conference on the Digital Humanities. University of Alberta. April 5-6, 2013.}

\cvitem{[71]}{ i O'Donnell, Daniel Paul. “New Business Models for Open Access Publication in the Humanities: The Lethbridge Journal Incubator.” Beyond the PDF2. Amsterdam. 19 March, 2013.}

\cvitem{[70]}{ r O'Donnell, Daniel Paul. “Far From the Maddening Crowd: Digital Projects and the Ethics of Popular Knowledge Mobilization.” Daniel O’Donnell, Catherine Karkov, and Heather Hobma. Havana. 12 December, 2012.}

\cvitem{[69]}{ r O'Donnell, Daniel Paul. "Is there a text in this edition? On the implications of multiple media and immersive technology for the future of the ‘scholarly edition’." Daniel P. O’Donnell, James Graham, Catherine E. Karkov, and Roberto Rosselli del Turco (University of Torino), European Society for Textual Scholarship (ESTS), Amsterdam. 23 November, 2012.}

\cvitem{[68]}{ i O'Donnell, Daniel Paul. "The Lethbridge Journal Incubator: Aligning scholarly publishing with the teaching and research missions of a public university." Canadian Association of Learned Journals. Congress of the Federation of the Social Sciences and Humanities. Waterloo, Ontario. May 27, 2012.}

\cvitem{[67]}{ i O'Donnell, Daniel Paul. "Markup and Metadata: An introduction to the power of XML and related technologies in humanities research applications." Pre-conference Workshop. Society for Digital Humanities/Société pour l’étude des médias interactifs. Congress of the Federation of the Social Sciences and Humanities. Waterloo, Ontario. May 25, 2012.}

\cvitem{[66]}{ i O'Donnell, Daniel Paul. "abdah.org: Alberta Digital Arts and Humanities and Campus Alberta." Presentation to Campus Alberta Arts, Social Sciences, and Humanities. University of Calgary. May 11, 2012.}

\cvitem{[65]}{ i O'Donnell, Daniel Paul. "Move Over: Learning to Read (and Write) with Novel Technology." MARCS: The Medieval and Renaissance Cultural Studies Research Group. University of  Calgary, March 15, 2012.}

\cvitem{[64]}{ r O'Donnell, Daniel Paul. "'Nor doubted once': Editing Text and Context." INKE Research Foundations For Understanding Books And Reading In A Digital Age Text And Beyond. Ritsumeikan University, Kyoto, Japan. November 18th, 2011.}

\cvitem{[63]}{ i O'Donnell, Daniel Paul. “The Medieval Academy's Digital Initiatives.” Invited Lecture. Medieval Electronic Scholarly Alliance (MESA). Mellon Workshop, Baltimore, May 2, 2011.}

\cvitem{[62]}{ i O'Donnell, Daniel Paul. “Digital Humanities and 'The Digital Humanities', Or, Should a Digital Humanities Center be Concerned with Word Processing.” Invited Lecture. Texas A&M University. February 17, 2011.}

\cvitem{[61]}{ r O'Donnell, Daniel Paul. “What Comes Between: Editing Context.” 7th Annual Conference. European Society for Textual Scholarship. Pisa, Italy. November 2010.}

\cvitem{[60]}{ r O'Donnell, Daniel Paul. “What is a Markup Language, Really? A Generic and Modular Approach to Understanding Markup Semantics” TEI Annual Conference and Members' Meeting, Zadar, Croatia. November 2010.}

\cvitem{[59]}{ r O'Donnell, Daniel Paul. “And everybody goes ahh: thinking outloud about interoperability.” Response paper. Digitized Collections of Medieval Manuscripts}

\cvitem{[58]}{ A Workshop on Uses and Interoperation. Paris, January 14-15, 2010. http://lib.stanford.edu/DMSS (https://lib.stanford.edu/files/ODonnellMellon1Smaller.pdf)}

\cvitem{[57]}{ i O'Donnell, Daniel Paul. "TEI: What and Why?" Invited Lectures and Workshop. Textual heritage and modern information technologies. Izhevsk, Russia. October 11-15, 2009.}

\cvitem{[56]}{ i O'Donnell, Daniel Paul. "Exploiting TEI Markup." Invited Lectures and Workshop. Textual heritage and modern information technologies. Izhevsk, Russia. October 11-15, 2009.}

\cvitem{[55]}{ i O'Donnell, Daniel Paul. "Cædmon's Hymn: Project Management and Development." Invited Lectures and Workshop. Textual heritage and modern information technologies. Izhevsk, Russia. October 11-15, 2009.}

\cvitem{[54]}{ i O'Donnell, Daniel Paul. "Sugar and Spice and... Sausage filling. What the TEI is made of." Invited Lecture.     Early Chán Manuscripts among the Dūnhuáng Findings– Resources in the Mark-up and Digitalization of Historical Texts. Oslo, September 30, 2009.}

\cvitem{[53]}{ i O'Donnell, Daniel Paul. "Are you sure we're not in Kansas any more, Dorothy? Domain knowledge and the future of the Digital Humanities." (Invited Lecture and Workshop). University of South Carolina September 24, 2009.}

\cvitem{[52]}{ i O'Donnell, Daniel Paul. "Sugar and Spice and... Sausage filling. What the TEI is made of" (Institute Lecture). Invited Lecture. Digital Humanities Summer Institute, University of Victoria. June 6, 2009.}

\cvitem{[51]}{ i O'Donnell, Daniel Paul. "We are Family: Digital Medievalist as Community of Practice." Medieval Studies and New Media / Les Médiévistes et les Nouveaux Media. ENS-Lyon. March 31, 2009.}

\cvitem{[50]}{ i O'Donnell, Daniel Paul. "Mind the Gap: Editing the spaces between objects in a post print world" (Keynote). Beyond Analogue: Current Graduate Research in Humanities Computing. University of Alberta. February 13, 2009.}

\cvitem{[49]}{ i O'Donnell, Daniel Paul. "Mind the Gap: Representing the Relationships among Constituents in a Multi-Object Digital Edition" (Keynote). Incontri di Filologia digitale. Università degli Studi di Verona, January 15, 2009.}

\cvitem{[48]}{ i O'Donnell, Daniel Paul. "Standoff Markup." Invited participant. Roundtable. CASTA. University of Saskatchewan. October 18, 2008.}

\cvitem{[47]}{ r O'Donnell, Daniel Paul. "'Murder to dissect'?: Digitisation as a Theory of the Text." SDH/SÉMI 2007, University of Saskatchewan. May 29, 2007.}

\cvitem{[46]}{ r O'Donnell, Daniel Paul. "The Visionary Cross: An Experiment in the Multimedia Edition." First author with Catherine Karkov, James Graham, Wendy Osborn, Roberto Rosselli Del Turco (read by Dorothy Porter, University of Kentucky). Digital Humanities 2007. University of Illinois, Urbana-Champaign. June 5, 2007.}

\cvitem{[45]}{ i O'Donnell, Daniel Paul. “We are family: the economics of best practice.” Invited Lecture. TEI Members Meeting, Victoria BC. October 27, 2006.}

\cvitem{[44]}{ r O'Donnell, Daniel Paul. “How Digital must a digital edition be?” 41st International Congress on Medieval Studies. University of Western Michigan. May 7, 2006.}

\cvitem{[43]}{ i O'Donnell, Daniel Paul. “Why should I write for your Wiki.” Renaissance Society of America. San Francisco. March, 2006.}

\cvitem{[42]}{ O'Donnell, Daniel Paul. “Using Electronic Media to Improve Efficiency and Intelligibility in Teaching and Researching the Middle Ages” [Satirical lecture]. Societas Fontibus Historiæ medii Aevii Inveniendis, vulgo dicta “The Pseudo Society”. 40th International Congress on Medieval Studies, Western Michigan University (Kalamazoo), May 7, 2005.}

\cvitem{[41]}{ r O'Donnell, Daniel Paul. “Back to the future: what electronic editors can learn from print editions of texts in multiple versions.” European Society for Textual Studies Conference. Alicante, Spain. November 24-25, 2004.}

\cvitem{[40]}{ r O'Donnell, Daniel Paul. “Best Practice in the Production of Digital Resources for Medievalists: Theory and Application”. Thirty-ninth International Congress on Medieval Studies (Kalamazoo). May, 2004.}

\cvitem{[39]}{ r O'Donnell, Daniel Paul. “Best Practice in the Production of Digital Resources for Medievalists: Project Design, Management, and Implementation”. Thirty-ninth International Congress on Medieval Studies (Kalamazoo). May, 2004.}

\cvitem{[38]}{ i O'Donnell, Daniel Paul. “Now What?: The Digital Medievalist Project and the Discovery of Best Practice.” Invited Lecture. SSHRC/University of Calgary ITST Summer Institute. May 26, 2004.}

\cvitem{[37]}{ i O'Donnell, Daniel Paul. “The Electronic Cædmon's Hymn: A Single Scholar, Multiple Text Electronic Edition and Archive.” Invited Lecture. SSHRC/University of Saskatchewan ITST Conference, May 15, 2004.}

\cvitem{[36]}{ i O'Donnell, Daniel Paul. “The Text Encoding Initiative: A Theoretical Standard for the Encoding of Electronic Texts.” Invited Lecture. SSHRC/University of Saskatchewan ITST Conference, May 15, 2004.}

\cvitem{[35]}{ r O'Donnell, Daniel Paul. “Poetry, Prose, and Book History: A way forward in debates about scribal literacy in Anglo-Saxon England?” Poetry and Prose: Intersections (to 1100): Methods and Approaches (Organisers: Carin Ruff and Elizabeth M. Tyler). Thirty-ninth International Congress on Medieval Studies (Kalamazoo). May 8, 2004.}

\cvitem{[34]}{ r O'Donnell, Daniel Paul. “Texts and the Single Scholar: Is the morning after worth the night before?” [Lecture on electronic project management]. Thirty-eighth International Congress on Medieval Studies (Kalamazoo). May 8, 2003.}

\cvitem{[33]}{ O'Donnell, Daniel Paul. “'Vade retro me Satana': What the Norton Anthology of Poetry gets Wrong on Page 1.” Department of English Colloquium series. University of Lethbridge. November 28, 2002.}

\cvitem{[32]}{ O'Donnell, Daniel Paul. “The Cædmon Code: Some Problems with Numerical and Geometrical Patterning in an Early Medieval Text.” Working Papers in the Humanities Colloquium. University of Lethbridge. April 4, 2002.}

\cvitem{[31]}{ O'Donnell, Daniel Paul. “‘Is that your final answer?’ Reconstructing Cædmon’s Hymn in a Post-Modern Age.” Germanic Philology Session, MLA Annual Meeting. Washington D.C., December 27, 2000. Also delivered the Department of English Research Colloquium, January, 2001.}

\cvitem{[30]}{ O'Donnell, Daniel Paul. “Text and Context: Generic Factors affecting Scribal Performance in the Transmission of Old English Verse.” Department of English Research Colloquium. University of Lethbridge. November 15, 2000.}

\cvitem{[29]}{ i O'Donnell, Daniel Paul. “The Editor Function, or I Know More about Cædmon’s Hymn than you do, Nyah, Nyah!” Inaugural Lecture, Humanities Computing Series, University of Calgary, June 15, 2000.}

\cvitem{[28]}{ r O'Donnell, Daniel Paul. “Reading Bede Reading Cædmon: Understanding a Critical Miracle.” Thirty-fifth International Congress on Medieval Studies (Kalamazoo). May 8, 2000. This was a thoroughly revised and abridged version of my March 1999 Departmental lecture and April 2000 lecture at Queen’s.}

\cvitem{[27]}{ i O'Donnell, Daniel Paul. “Reading Bede Reading Cædmon: Bede’s Historia ecclesiastica as Source and Source of Interpretation for Cædmon’s Hymn.” Invited Lecture. Department of English, Queen’s University. April 10, 2000.}

\cvitem{[26]}{ O'Donnell, Daniel Paul. “The Editor Function: Form, Content, and Editorial Theory in Editing Cædmon’s Hymn.” Department of English Research Colloquium. University of Lethbridge. February 2, 2000. This was a revised version of my May 9, 1999 invited lecture at the Kalamazoo Medieval Studies Conference.}

\cvitem{[25]}{ r O'Donnell, Daniel Paul. “The Editor Function: Form, Content, and Editorial Theory in Editing Cædmon’s Hymn.” Commissioned Lecture. Thirty-fourth International Congress on Medieval Studies (Kalamazoo). May 9, 1999.}

\cvitem{[24]}{ O'Donnell, Daniel Paul. “Reading Bede Reading Cædmon: Bede’s Historia ecclesiastica as Source and Source of Interpretation for Cædmon’s Hymn.” Department of English Research Colloquium. University of Lethbridge. March 1999.  A lightly revised version of this paper was delivered as an invited lecture, April 10, 2000 at Queen’s University.}

\cvitem{[23]}{ i O'Donnell, Daniel Paul. “Fish and Fowl: Generic Expectations and the Relationship between the Old English Phoenix poem and Lactantius’s de ave phoenice.” Germania Latina IV. Groningen, The Netherlands, July 1998.}

\cvitem{[22]}{ O'Donnell, Daniel Paul. “What Anne Meant: Generic Instability and the Transmission of Anne Frank’s Diary.” Department of English Research Colloquium. University of Lethbridge. March 1998.}

\cvitem{[21]}{ r O'Donnell, Daniel Paul. “A New Theory of Poetic Textual Transmission.” Delivered at: “Anglo-Saxon Studies in the Twentieth Century.” International Society of Anglo-Saxonists Conference (Palermo, Italy). July 11, 1997.}

\cvitem{[20]}{ r O'Donnell, Daniel Paul. “The Text of Cædmon’s Hymns.” Delivered at: “Focusing on Editorial Scholarship at the Century’s End.” MLA Convention, Washington D.C. December 28, 1996.}

\cvitem{[19]}{ r O'Donnell, Daniel Paul. “‘Transitional Literacy’ and the Poems of the Anglo-Saxon Chronicle: Context as Counter-evidence.” Medieval Chronicle Conference. Rijksuniversiteit Utrecht/Driebergen. July 13, 1996.}

\cvitem{[18]}{ i O'Donnell, Daniel Paul. “Ends and Means: Manuscript Context and Scribal Accuracy in the Copying of Anglo-Saxon Poetry.” Invited Lecture. Department of English, Trinity College Dublin. March 19, 1996.}

\cvitem{[17]}{ i O'Donnell, Daniel Paul. “The Spirit and the Letter: The Use of the Dramatic in Old Frisian Legal Writing.” Twenty-ninth International Congress on Medieval Studies (Kalamazoo). May 4, 1994.}

\cvitem{[16]}{ r O'Donnell, Daniel Paul. “Beinecke MS 594: A Second Look at a Well-Known Nominale.” Early Book Society Conference. Trinity College Dublin. 1991.}

\cvitem{[15]}{ O'Donnell, Daniel Paul. “The OE ‘Phoenix’, Lactantius’s ‘De ave phoenice’ and the Science of Allegory,” Harvard-Yale Graduate Student Colloquium. 1991.}

\cvitem{[14]}{ Other Lectures}

\cvitem{[13]}{ O’Donnell, Daniel Paul. 2022. “Solidarity: Some thoughts on the recent strike and lockout at the U of Lethbridge.” Northern Alberta Institute of Technology. April 21.}

\cvitem{[12]}{ O’Donnell, Daniel Paul. 2021. “You can’t handle the truth! Academic freedom and University Governance. Mount Royal University Faculty Association. March 5.}

\cvitem{[11]}{ O’Donnell, Daniel Paul, Paul Hayes, Andrea Amelinckx, Annabree Fairweather, Terry Sway, Rumi Graham. 2018. I am he as you are he and you are me and we are all together: Communicating with professional know-it-alls. CAFA Labour Conference. March 9.}

\cvitem{[10]}{ O'Donnell, Daniel Paul 2018. '“Know your song well before you start singin’”: Negotiating when “black is the color” and “none is the number”'. CAUT Bargaining Roundtable.}

\cvitem{[9]}{ O'Donnell, Daniel Paul, Paul Hayes. 2017. 'Understanding Lockout/ Strike in the Post Secondary Sector under the Alberta Labour Relations Code'. ULFA. Lethbridge. Canada. https://zenodo.org/record/1045544.}

\cvitem{[8]}{ General Interest and Undergraduate Lectures}

\cvitem{[7]}{ “’The cipher manuscript’: Hoaxes as a book history problem.” Invited guest lecture. English 4400. Book History. Nov. 6, 2021. https://doi.org/10.5281/zenodo.5651109.}

\cvitem{[6]}{ “Beowulf: Manuscripts, Pronunciation, Grammar, and Metre.” Invited guest lecture. English 1900. University of Lethbridge. Sept. 17, 2014.}

\cvitem{[5]}{ “The Importance of Being Earnest: Coding Problems and Solutions.” Invited Guest Lecture, English 517/607 (Graduate Humanities Computing), University of Calgary. June 15, 2000.}

\cvitem{[4]}{ “Tolkien’s Elvish and Other Constructed Languages.” Invited Lecture, English 3700 (Children’s Literature), University of Lethbridge. February 7, 2000.}

\cvitem{[3]}{ “Anne, Otto, and the Neo-Nazis. The (Im)morality of Reading a Holocaust Diary.” Invited Lecture, Arts and Science 1000 (Liberal Arts), University of Lethbridge. November 2001; September 2000; November, 1999.}

\cvitem{[2]}{ “What Anne Meant: Generic Instability and the Transmission of Anne Frank’s Diary.” Invited Lecture, Arts and Science 1000 (Liberal Arts), University of Lethbridge. November 2001; September 2000; November, 1999.}

\cvitem{[1]}{ “Beowulf and its Place in History.” Worker’s Educational Association Weekend School at Horncastle. February, 1997.}