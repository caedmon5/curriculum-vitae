\documentclass[11pt,letterpaper,sans]{moderncv}

\moderncvstyle{casual}
\moderncvcolor{blue}
\usepackage[scale=0.85]{geometry}

% Prevent orphaned headings across all standard and moderncv commands
\usepackage{etoolbox}
\usepackage{needspace}

% Standard LaTeX sections
\pretocmd{\section}{\Needspace{5\baselineskip}}{}{}
\pretocmd{\subsection}{\Needspace{3\baselineskip}}{}{}
\pretocmd{\subsubsection}{\Needspace{3\baselineskip}}{}{}

% moderncv-specific commands
\pretocmd{\cvsection}{\Needspace{5\baselineskip}}{}{}
\pretocmd{\cvsubsection}{\Needspace{3\baselineskip}}{}{}
\pretocmd{\cvsubsubsection}{\Needspace{3\baselineskip}}{}{}

\newcommand{\cvheading}[2]{%
  \Needspace{#1}%
  \section{#2}
}

\usepackage[backend=biber, style=authoryear-comp, sorting=ydnt, maxbibnames=99, dashed=false]{biblatex}
\addbibresource{articles_and_chapters.bib}

% Encoding for special characters
\usepackage[T1]{fontenc}
\usepackage[utf8]{inputenc}
\usepackage[american]{babel}

% Set spacing between entries
\setlength{\bibitemsep}{0.15em}

% Custom bibliography environment to align with section and add year in margin
\defbibenvironment{bibliography}
  {\list
     {\printfield{year}} % Year shown in left margin
     {\setlength{\labelwidth}{3em}
      \setlength{\leftmargin}{\labelwidth}
      \setlength{\labelsep}{1em}
      \setlength{\itemsep}{0.15em}
      \setlength{\parsep}{0pt}
      \setlength{\topsep}{0pt}
      \setlength{\partopsep}{0pt}
      \setlength{\itemindent}{0pt}
      \setlength{\listparindent}{0pt}
      \renewcommand*{\makelabel}[1]{\hfill\textbf{##1}}}}
  {\endlist}
  {\item}

\name{Daniel Paul}{O'Donnell}
\title{Curriculum Vitae}
\email{daniel.odonnell@uleth.ca}
\social[github]{caedmon5}
\social[orcid]{https://orcid.org/0000-0002-0127-4893}
\social[zenodo]{https://zenodo.org/communities/dpodrepository}
\social[home]{https://uleth.ca/dspace/handle/10133/3557}

\begin{document}

\makecvtitle

\section{Education}
\cvitem{\textbf{Summary}}{O'Donnell completed his undergraduate studies at the University of Toronto in English and Medieval Latin, followed by an MA and PhD at Yale University in English. His doctoral research focused on variation in Old English poetry, supervised by Fred C. Robinson. \vspace{1em}}

\cventry{1996}{Ph.D. in English}{Yale University}{}{Andrew W. Mellon Fellow; SSHRC Doctoral Fellow; Yale University Fellow \vspace{0.25em} \newline{}\textbf{Dissertation:} \emph{\href{https://doi.org/10.5281/zenodo.1171976}{Manuscript Variation in Multiple Recension Old English Poetic Texts: The Technical Problem and Poetical Art}} \newline{}\textbf{Supervisor:} Fred C. Robinson \newline{}\textbf{Reviewed in} \emph{Linguistica e Filologia} 9 (1999): 156}{}
\vspace{0.5em}
\cventry{1991}{M.A. in English}{Yale University}{}{Andrew W. Mellon Fellow; Yale University Fellow}{}
\vspace{0.5em}
\cventry{1989}{B.A. in English Language and Literature (with Distinction)}{St. Michael's College, University of Toronto}{}{Various Prizes \vspace{0.25em}\newline{}\textbf{Specialization:} English Language and Literature\newline{}\textbf{Minor:} Medieval Latin (requirements also met for minor in Celtic Studies)}{}
\vspace{1em}

\section{Languages}
\cvitem{\textbf{Summary}}{O'Donnell works in a range of modern and historical languages relevant to medieval studies and digital scholarship. These include English (native), Dutch (spoken and read), French (reading and conversational), and German (reading), as well as a variety of medieval and classical languages, such as Old English, Old Norse, Latin, Old Frisian, and Gothic.}

\vspace{1em}

\cvitem{Modern}{
\begin{itemize}
  \item English (native)
  \item Dutch (preparing for CERF C2 certification)
  \item French (preparing for CERF B2 certification)
  \item German (reading)
\end{itemize}
}
\vspace{0.25em}
\cvitem{Historical}{Old English, Old Norse, Latin, Old Frisian, Gothic}
\vspace{1em}

\section{Academic Positions}
\cvitem{\textbf{Summary}}{Since 1997, O’Donnell has held a continuing faculty position at the University of Lethbridge, where he currently holds the rank of Professor. He has also taught at Louisiana State University, the University of York, and Yale University, and has held adjunct and affiliate roles at the University of Saskatchewan and within the University
Library. At the University of Toronto, he was the first undergraduate to hold a research position at the Dictionary of Old English.}

\vspace{1em}
\cventry{2020--}{Affiliate Member}{Prentice Institute for Global Population, University of Lethbridge}{}{}{}
\cventry{2019--2024}{Adjunct Professor}{College of Postgraduate and Postdoctoral Studies, University of Saskatchewan}{}{}{}
\cventry{2018}{Professor of English (Offered and Declined)}{University of Saskatchewan}{}{}{}
\cventry{2015--}{Associate Member}{University Library Academic Staff, University of Lethbridge}{}{}{}
\cventry{2010--}{Professor of English (Tenured)}{University of Lethbridge}{}{}{}
\cventry{2002--2010}{Associate Professor of English (Tenured)}{University of Lethbridge}{}{}{}
\cventry{1997--2002}{Assistant Professor of English}{University of Lethbridge}{}{}{}
\cventry{1997}{Tutor}{Department of English and Related Literatures, University of York (UK)}{}{}{}
\cventry{1997}{Tutor}{Workers’ Educational Association (UK)}{}{}{}
\cventry{1994--1995}{Visiting Assistant Professor}{Department of English, Louisiana State University}{}{}{}
\cventry{1991--1992}{Teaching Fellow}{Department of English, Yale University}{}{}{}
\cventry{1987--1989}{Research Assistant}{Dictionary of Old English, University of Toronto}{}{}{}

\vspace{1em}

\section{Academic Leadership}
\cvitem{\textbf{Summary}}{O'Donnell has led multiple academic and professional organisations, including CAFA, ULFA, Force11, GO::DH, and the TEI. He has chaired major digital infrastructure and policy boards, founded scholarly and advocacy groups, and served as department chair at Lethbridge. In these roles, he has consistently taken leadership during moments of structural transition and policy change.}

\vspace{1em}
\cvitem{2005--2008, 2023--}{\textbf{Department Chair}, Department of English, University of Lethbridge. \vspace{0.15em}\newline{}\textit{Led department through curriculum overhaul, review, and hiring plan during institutional crisis; implemented new governance model}.}
\vspace{0.25em}
\cvitem{2024--2025}{\textbf{Past President}, Confederation of Alberta Faculty Associations (CAFA). \vspace{0.15em}\newline{}\textit{Continued strategic advocacy. Helped establish quarterly meetings with Deputy Minister and Staff at Ministry of Advanced Education (Alberta). Co-presenter to "Expert Panel on Post-Secondary Institution Funding and Alberta’s Competitiveness (Mintz Panel)}.}
\vspace{0.25em}
\cvitem{2023--2024}{\textbf{President}, CAFA. \vspace{0.15em}\newline{}\textit{Assumed role immediately following loss of major member (2/3 of income). Rebuilt organisation and expanded connections to other post-secondary and civil society organisations in province and western Canada as well as nationally. Redesigned public and government relations practices, resulting in improved press-coverage and regular meetings with government officials. Build successful inter-sectoral lobbying effort on Bill 18 (with Rural Municipalities of Alberta and University of Alberta. Helped found Western Regional Council.} \vspace{0.25em}}
\vspace{0.25em}
\cvitem{2023--2024}{\textbf{Past President}, University of Lethbridge Faculty Association (ULFA). \vspace{0.15em}\newline{}\textit{Supported transition to new leadership. Advised on grievance resolution, governance, member relations, communications, and bargaining. Negotiated MoU extending ULFA/U of L collective agreement to cover faculty at ULethbridge International College Calgary (UICC), a new Navitas/University of Lethbridge collaboration. This is the first agreement nationally to extend Collective Agreement protections completely to instructors in Navitas-led college}.}
\vspace{0.25em}
\cvitem{2021--2023}{\textbf{President}, ULFA. \vspace{0.15em}\newline{}\textit{Led Association communications and preparations for job action during two years of negotiations. Oversaw Association during six-week job action in February and March of 2022 (92\% strike vote, 91\% ratification). Led restoration of management-union relations post-strike}.}
\vspace{0.25em}
\cvitem{2020--2021}{\textbf{Vice-President}, ULFA. \vspace{0.15em}\newline{}\textit{Assisted president in major grievance cases while leading bargaining}.}
\vspace{0.25em}
\cvitem{2018--2021}{\textbf{Chief Spokesperson and Bargaining Chair}, ULFA. \vspace{0.15em}\newline{}\textit{Negotiated multiple collective agreements (including Navitas MOU); launched and secured approval for job action fund}.}
\vspace{0.25em}
\cvitem{2018--2019}{\textbf{President}, Force11 (Future of Research Communications and E-scholarship). \vspace{0.15em}\newline{}\textit{Oversaw launch of FORCE11 Scholarly Communication Institute (FSCI); expanded global FAIR and Open Science policy advocacy)}.}
\vspace{0.25em}
\cvitem{2013--2017}{\textbf{Vice-President}, Force11. \vspace{0.15em}\newline{}\textit{Built global partnerships in scholarly communication and infrastructure}.}
\vspace{0.25em}
\cvitem{2015--2017}{\textbf{Founding Director}, Lethbridge Centre for the Study of Scholarly Communication. \vspace{0.15em}\newline{}\textit{Created institutional centre supporting interdisciplinary research and publication innovation in Open Science and scholarly communication}.}
\vspace{0.25em}
\cvitem{2012--2015}{\textbf{Founding Chair}, Global Outlook::Digital Humanities (GO::DH). \vspace{0.15em}\newline{}\textit{Founded equity-centred digital humanities network credited with leading the "Global Turn" in Digital Humanities Research. Initiated multilingual and regional DH programs}.}
\vspace{0.25em}
\cvitem{2010--2013}{\textbf{Co-President}, Canadian Society for Digital Humanities (CSDH/SCHN). \vspace{0.15em}\newline{}\textit{Coordinated national DH strategy and membership growth}.}
\vspace{0.25em}
\cvitem{2010--}{\textbf{Editor-in-Chief}, \textit{Digital Studies/Les champs numériques} (DSCN). \vspace{0.15em}\newline{}\textit{Assumed editorial leadership of national journal in DH studies at the end of volume 2. Established alternate funding model to preserve Open Access status. Oversaw approval for membership in the Open Library of the Humanities (OLH)}.}
\vspace{0.25em}
\cvitem{2009--2012}{\textbf{Chair}, Digital Initiatives Advisory Board, Medieval Academy of America (MAA). \vspace{0.15em}\newline{}\textit{Guided MAA infrastructure strategy; advised on journal and archive digitisation}.}
\vspace{0.25em}
\cvitem{2006--2011}{\textbf{Chair and CEO}, Text Encoding Initiative (TEI). \vspace{0.15em}\newline{}\textit{Restructured TEI governance; launched member-led consortium; published revised TEI Guidelines}.}
\vspace{0.25em}
\cvitem{2003--2009}{\textbf{Founding Director}, Digital Medievalist (DM). \vspace{0.15em}\newline{}\textit{Established pioneering digital community of practice for medievalists working with digital media. Built infrastructure and journal}.}
\vspace{1em}

\section{Current Major Projects}
\cvitem{\textbf{Summary}}{O’Donnell is Principal Investigator (PI) on several research initiatives focused on scholarly communication, data infrastructure, and academic governance, and early medieval culture. These projects are supported by internal and external funding from various programmes. The training of Highly Qualified Personnel (HQP) is an important element in all these projects.}

\vspace{1em}
\cvitem{Resistance to Data}{Humanities Approaches to Research Classification and Infrastructure (2022--) \vspace{0.15em}\newline{}\textbf{Funding:} SSHRC Insight Grant 2025--2029, PI, \$300,000; SSHRC PDG 2022--2025, PI, \$200,000. \newline{}\textit{Umbrella initiative investigating epistemic and institutional resistance to structured data in the humanities. Builds on the earlier projects} Good Things Come in Small Packages \textit{and} Humanities Data Inquiry (HDI) \textit{projects}.}
\vspace{0.25em}
\cvitem{\#safespaces?}{Academic Freedom and Inclusivity in an Age of Polarisation (2024--) \vspace{0.15em}\newline{}\textbf{Funding:} SSHRC Insight Development Grant (applied for), PI. \newline{}\textit{Examines how political polarisation and campus speech discourses intersect with evolving understandings of academic freedom}.}
\vspace{0.25em}
\cvitem{Lethbridge Journal Incubator}{Using Open Access publication to provide work-integrated-learning opportunities to graduate students (2012--)  \vspace{0.15em}\newline{}\textbf{Funding:} SSHRC Aid to Scholarly Journals (PI); School of Graduate Studies; School of Graduate Studies, Strategic Opportunities Fund (SOF). \newline{}\textit{Training and mentorship program in scholarly publishing. Provides graduate students with editorial and infrastructure experience}.}
\vspace{0.25em}
\cvitem{Visionary Cross}{Modelling the Ruthwell Cross, Bewcastle Cross, and Brussels Cross (2005--). \vspace{0.15em}\newline{}\textbf{Funding:} Chinook Summer Student Grant 2025 (\$6,000); Mitacs Global Interns; SSHRC Insight Grant 2014–2017 (\$282,190); SSHRC SRG 2005–2008. \newline{}\textit{A digital edition and visualization of the Ruthwell Cross and related artifacts. Combines philology, archaeology, and imaging. Includes multi-format outputs and graduate training}.}
\vspace{1em}

\section{Funding and Prizes}
\subsection*{External Research Grants and Prizes}
\cvitem{\textbf{Summary}}{O’Donnell has received approximately CA\$4 million in research funding as Principal Investigator and Co-applicant from external agencies since 2000, including approx. CA\$1.4 million as PI (the University of Lethbridge's top-earner from SSHRC-funded programmes). His work spans scholarly communication, early Medieval English studies, research infrastructure, and innovation in academic publishing.}
\vspace{1em}
\cvitem{2025}{\textbf{SSHRC Impact Prize (Connections)} (780-2025-00021), Nominee. \$50,000 (under review)}
\cvitem{2025--2027}{\textbf{SSHRC Insight Development Grant}, \emph{\#safespaces?: Academic Freedom and Inclusivity in an Age of Polarisation}, PI, \$78,000 (under review)}
\cvitem{2025--2029}{\textbf{SSHRC Insight Grant} (435-2025-0705), \emph{Resistance to Data: Understanding Data Use, Data Management, and Data Infrastructure in the "Traditional" Humanities through Historical, Comparative, and Ethnographic Study}, PI, \$300,000}
\cvitem{2021--2026}{\textbf{SSHRC Insight Grant}, \emph{Canterbury Tales Project}, Co-applicant, \$340,000}
\cvitem{2021--2023}{\textbf{SSHRC Partnership Development Grant} (890-2020-0095), \emph{Good Things Come in Small Packages: A Grassroots Community of Practice for Open and FAIR Humanities Data Practices}, PI, \$200,000}
\cvitem{2022--2025}{\textbf{SSHRC Aid to Scholarly Journals} (651-2021-0150), \emph{Digital Studies / Les champs numériques}, PI, \$30,000}
\cvitem{2019--2022}{\textbf{SSHRC Aid to Scholarly Journals} (651-2018-0062), \emph{Digital Studies / Les champs numériques}, PI, \$90,000}
\cvitem{2020--2021}{\textbf{SSHRC Connections Grant} (611-2019-0499), \emph{Canada-LATAM Workshop on Open and Inclusive Access to research}, PI, \$25,000}
\cvitem{2020--2021}{\textbf{Sloan Foundation} (G-2020-13999), \emph{REPO: Reimagining Education Practices for Open. Developing Open Science during the COVID emergency: A community-based investigation and support network}, PI, US\$50,000}
\cvitem{2017--2019}{\textbf{SSHRC Partnership Development Grant} (890-2016-0081), \emph{Future Commons: Transforming Scholarly Communication through Collective Action}, PI, \$200,000}
\cvitem{2017--2019}{\textbf{CFI John R. Evans Leadership Fund} (JELF) (32819), \emph{What Goes Around: The Visionary Cross Digital Library}, PI, \$86,937}
\cvitem{2015--2019}{\textbf{SSHRC Insight Grant} (435-2015-1119), \emph{What Goes Around: Editing the Anglo-Saxon Visionary Cross Cultural Matrix}, PI, \$233,224}
\cvitem{2017--2018}{\textbf{Mellon Foundation}, \emph{Reading Peer Review}, Co-PI, US\$99,000}
\cvitem{2014--2017}{\textbf{SSHRC Aid to Scholarly Journals} (651-2014-0138), \emph{Digital Studies}, PI, \$46,225}
\cvitem{2013--2017}{\textbf{SSHRC Insight Grant}, \emph{Canterbury Tales Project Phase 2}, Co-applicant, \$471,000}
\cvitem{2015}{\textbf{Helmsley Charitable Trust} (via Force11), Drafting applicant, US\$424,000}
\cvitem{2015}{\textbf{Mitacs GlobalLink}, \emph{Visionary Cross}, In-kind}
\cvitem{2014--2015}{\textbf{Gordon and Betty Moore Foundation}, \emph{Force11}, Co-applicant, US\$150,000}
\cvitem{2014--2015}{\textbf{GRAND Startup Grant} (G-CI-14-LB-01), \emph{DigiCultH: Engaging with Digital Cultural Heritage Objects}, Co-PI, \$9,640}
\cvitem{2010--2014}{\textbf{SSHRC Standard Research Grant (SRG)} (410-2010-1474), \emph{Crossroads: Editing the Visionary Cross Matrix in Anglo-Saxon England}, PI, \$62,430}
\cvitem{2008}{\textbf{Mellon Foundation} (GP1.2008), \emph{TEI Tite: Creating a Benefit of Membership to Support Standards Development}, PI, US\$30,723}
\cvitem{2005--2006}{\textbf{SSHRC Image, Text, Sound, and Technology (ITST)} (849-2003-0003), \emph{The Digital Medievalist Project: A Community of Practice Network for Image, Text, Sound and Technology Research}, PI, \$27,490}
\vspace{1em}

\subsection*{Awards, Fellowships, and Honours}
\cvitem{\textbf{Summary}}{O’Donnell received honorable mention in the MLA prize competition for the most-distinguished scholarly edition, the first time a digital critical edition had been recognised in this competition. As a graduate student, he was a Mellon Fellow in the Humanities (1989--1991; 1993--1994); a SSHRC Doctoral Fellow (1992--1994); Yale University Fellow (1989--1993); and UCLA Fellow (declined). He was University of Lethbridge Teaching Award Nominee in 2015--2016 and 2016--2017 and won several undergraduate awards.}
\vspace{1em}
\cvitem{2015--2017}{University of Lethbridge Teaching Award, Nominee}
\cvitem{2007}{Honourable Mention, MLA Prize for a Scholarly Edition (Cædmon’s Hymn)}
\cvitem{1993--1994}{Mellon Fellowship in the Humanities (Dissertation Fellowship), US\$11,000}
\cvitem{1992--1994}{SSHRC Doctoral Fellowship, CA\$14,000/year}
\cvitem{1992--1993}{Yale University Fellowship, US\$16,000 + tuition}
\cvitem{1989--1993}{UCLA University Fellowship, US\$36,000 + tuition (declined)}
\cvitem{1989--1991}{Mellon Fellowship in the Humanities, US\$22,000 + tuition}
\cvitem{1989--1991}{Yale University Fellowship, US\$16,000 + tuition (declined)}
\cvitem{1988--1989}{C.L. Burton In-course Scholarship, St. Michael’s College, University of Toronto, CA\$1,500 (declined)}
\cvitem{1987--1988}{In-course Scholarship, St. Michael’s College, University of Toronto, CA\$1,500 (declined)}

\cvheading{10\baselineskip}{Publications}

\cvitem{}{\textit{r = refereed; i = invited; s = student co-authors at time of writing (student authors marked with asterisk).}}

\cvsubheading{5\baselineskip}{Books and Editions}

\cvitem{[2]}{rs Eve, Martin Paul, Neylon, Cameron, \textbf{O'Donnell, Daniel Paul}, Moore, Samuel, Gadie, Robert*, Odeniyi, Victoria*, and Parvin, Shahina*. 2020. \textit{Reading Peer Review: PLOS ONE and Institutional Change in Academia}. Cambridge: Cambridge University Press. \textsc{doi}: \url{https://doi.org/10.1017/9781108783521}. ISBN: 9781108486637 (hardback); 9781108783521 (ebook).}
\cvitem{}{– \textbf{Review:} \textit{Public Understanding of Science} 31.7 (2022): 892–894.}

\cvitem{[1]}{r \textbf{O'Donnell, Daniel Paul}. 2005. \textit{Cædmon’s Hymn: A Multimedia Study, Edition and Archive}. SEENET A.8. Cambridge: Medieval Academy of America and D.S. Brewer. xxii + 261 pp. + CD-ROM. Internet reprint (2018): \url{https://caedmon.seenet.org/}. Codebase \textsc{doi}: \url{https://doi.org/10.5281/zenodo.1198856}.}
\cvitem{}{– \textbf{Prize:} 2007 MLA Distinguished Scholarly Edition Prize (honorable mention).}
\cvitem{}{– \textbf{Reviews:}  \textit{e-data\&research} 1 (2006): 1; \textit{Medium Aevum} 75 (2006): 356–357; \textit{Old English Newsletter} 40.1 (2006); \textit{Speculum} 82 (2007): 223–224; \textit{Journal of Ecclesiastical History} 58 (2007): 120–121; \textit{Early Medieval Europe} 15 (2007): 466–469; \textit{Textual Cultures} 2 (2007): 139–142; \textit{JEGP} 107.2 (2008): 248–251; \textit{Digital Medievalist} 5 (2009); \textit{Leeds Medieval Studies} 3 (2023).}


\cvsubheading{5\baselineskip}{Articles and Chapters (Traditional)}

\cvitem{[44]}{rs \textbf{O'Donnell, Daniel Paul}, Pafumi, Davide*, Onuh, Frank*, Khalid, AKM Iftekhar*, Pearce, Morgan*, and Bordalejo, Barbara. 2025. “Scarlet Cloak and the Forest Adventure: A preliminary study of the impact of AI on commonly used writing tools.” \textit{International Journal of Educational Technology in Higher Education} 22.6. \textsc{doi}: \url{https://doi.org/10.1186/s41239-025-00505-5}.}

\cvitem{[43]}{ri \textbf{O'Donnell, Daniel Paul}. 2021. “‘I heard he sang a good song’: Caedmon’s inspiration and medieval dream theory.” In: \textit{Sogni, visioni e profezie nella letteratura germanica medievale}, edited by Rosselli Del Turco, Roberto. Bibliotheca Germanica. Studi e testi 48. Allessandria: Edizioni del l’Orso, pp. 147–172.}

\cvitem{[42]}{r \textbf{O'Donnell, Daniel Paul}, Bliss, Heather, Genee, Inge, and Junker, Marie-Odile. 2020. “‘Credit Where Credit Is Due’: Authorship and Attribution in Algonquian Language Digital Resources.” \textit{IDEAH} 1.1. Author order: Alphabetical. \textsc{doi}: \url{https://doi.org/10.21428/f1f23564.3d64b2ed}.}

\cvitem{[41]}{r \textbf{O'Donnell, Daniel Paul}. 2020. “Critical Mass: The Listserv and the Early Online Community as a Case Study in the Unanticipated Consequences of Innovation in Scholarly Communication.” In: \textit{Digital Technology and the Practices of Humanities Research}, edited by Edmond, Jennifer. Cambridge: Open Book Publishers, pp. 184–206. \textsc{doi}: \url{https://doi.org/10.5281/zenodo.3633429}.}

\cvitem{[40]}{r \textbf{O'Donnell, Daniel Paul}. 2019. “All along the Watchtower: Intersectional diversity as a core intellectual value in the Digital Humanities.” In: \textit{Intersectionality in Digital Humanities}, edited by Bordalejo, Barbara and Risam, Roopika. Amsterdam: ARC. \textsc{doi}: \url{https://doi.org/10.5281/zenodo.3580235}.}

\cvitem{[39]}{rs \textbf{O'Donnell, Daniel Paul}, Esau, Paul*, Viejou, Carey*, Chow, Sylvia*, Dohms, Kimberly*, Firth, Steve*, McKinnon, Jarret*, Morrison, Dorethea*, Parsons, Reed*, Rieger, Courtney*, Spiric, Vanja*, Toth, Elaine*, Ueland, Kayla*, and Graham, Rumi. 2018. “‘Let’s Start a Journal!’: The Multidisciplinary Graduate Student Journal as Educational Opportunity.” \textit{The Journal of Electronic Publishing (JEP)} 21.1. Corresponding author. \textsc{doi}: \url{https://doi.org/10.3998/3336451.0021.109}.}

\cvitem{[38]}{i \textbf{O'Donnell, Daniel Paul}, Callieri, Matteo, Dellepiane, Marco, Karkov, Catherine, Porter, Dot, and Rosselli Del Turco, Roberto. 2018. “Archaeology in the Study: Scanning Anglo-Saxon Artifacts in the Visionary Cross Project.” \textit{Wiðowinde} 185.Spring, pp. 21–27. \textsc{doi}: \url{https://doi.org/10.5281/zenodo.1208167}.}

\cvitem{[37]}{rs \textbf{O'Donnell, Daniel Paul}, Viejou, Carey*, Chow, Sylvia*, Dohms, Kimberly*, Esau, Paul*, Firth, Steve*, and Graham, Rumi. 2018. “Zombie Journals: Designing a Technological Infrastructure for a Precarious Graduate Student Journal.” \textit{Scholarly and Research Communication} 9.2. Corresponding author. \textsc{doi}: \url{https://doi.org/10.22230/src.2018v9n2a296}.}

\cvitem{[36]}{r \textbf{O'Donnell, Daniel Paul}, Moore, Samuel*, Neylon, Cameron, Eve, Martin Paul, and Pattinson, Damian. 2017. “‘Excellence R Us’: University Research and the Fetishisation of Excellence.” \textit{Palgrave Communications} 3.January. Author order: Random. \textsc{doi}: \url{https://doi.org/10.1057/palcomms.2016.105}.}

\cvitem{[35]}{r Tennant, Jonathan P., Dugan, Jonathan M., Graziotin, Daniel Jacques Damien, Waldner, François, Mietchen, Daniel, Elkhatib, Yehia, Collister, Lauren B., and others. 2017. “A Multi-Disciplinary Perspective on Emergent and Future Innovations in Peer Review.” \textit{F1000Research} 6.1151. Author order: Based on date of first edit. \textsc{doi}: \url{https://doi.org/10.12688/f1000research.12037.2}.}

\cvitem{[34]}{r \textbf{O'Donnell, Daniel Paul}, Champieux, Robin, Kramer, Bianca, Bosman, Jeroen, Bruno, Ian, Buckland, Amy, Callaghan, Sarah, Chapman, Chris, Hagstrom, Stephanie, and Martone, MaryAnn E. 2016. “Finding the Principles of the Commons: A Report of the Force11 Scholarly Communications Working Group.” \textit{Collaborative Librarianship} 8.2. \url{http://digitalcommons.du.edu/collaborativelibrarianship/vol8/iss2/5}.}

\cvitem{[33]}{rs \textbf{O'Donnell, Daniel Paul}, Hobma, Heather*, Karkov, Catherine, Foster, Sally, Graham, James, Osborn, Wendy, Rosselli Del Turco, Roberto, Broatch, Robert, Broatch, Susan, Callieri, Marco, and Dellepiane, Matteo. 2016. “Modern impact on the fabric of the Ruthwell Cross.” \textit{OEN} 46.1. Corresponding author. \url{http://www.oenewsletter.org/OEN/issue/ruthwell.php}.}

\cvitem{[32]}{r \textbf{O'Donnell, Daniel Paul}, Kramer, Bianca, Bosman, Jeroen, Ignac, Marcin, Kral, Christina, Kalleinen, Tellervo, Koskinen, Pekko, Bruno, Ian, Buckland, Amy, Callaghan, Sarah, Champieux, Robin, Hagstrom, Stephanie, Martone, MaryAnn, and Murphy, Fiona. 2016. “Defining the Scholarly Commons - Reimagining Research Communication.” \textit{Research Ideas and Outcomes} 2.May, e9340. \textsc{doi}: \url{https://doi.org/10.3897/rio.2.e9340}.}

\cvitem{[31]}{i \textbf{O'Donnell, Daniel Paul}. 2016. “The Bird in Hand: Humanities Research in the Age of Open Data.” In: \textit{The State of Open Data}, edited by Figshare. London: Digital Science, pp. 34–35. \textsc{doi}: \url{https://doi.org/10.5281/zenodo.1470821}.}

\cvitem{[30]}{rs \textbf{O'Donnell, Daniel Paul}, Bay, Jessica*, Dering, Emma*, Gal, Matt*, Grandfield, Virgil*, Hobma, Heather*, and Singh, Gurpreet*. 2016. “The Third Academic Freedom.” \textit{Light on Teaching}, pp. 4–9. Corresponding author. \textsc{doi}: \url{https://doi.org/10.5281/zenodo.3596098}.}

\cvitem{[29]}{rs \textbf{O'Donnell, Daniel Paul}, Leoni, Chiara*, Callieri, Marco, Dellepiane, Matteo, Rosselli Del Turco, Roberto, and Scopigno, Roberto. 2015. “The Dream and the Cross: a 3D scanning project to bring 3D content in a digital edition.” \textit{Journal on Computing and Cultural Heritage}. \textsc{doi}: \url{https://doi.org/10.1145/2686873}.}

\cvitem{[28]}{r \textbf{O'Donnell, Daniel Paul}, Gil, Alex, Walters, Katherine, and Fraistat, Neil. 2015. “Only Connect: The Globalization of the Digital Humanities.” In: \textit{A New Companion to Digital Humanities}, edited by Schreibman, Susan, Siemens, Ray, and Unsworth, John. Wiley, pp. 493–510. \textsc{doi}: \url{https://doi.org/10.1002/9781118680605.ch34}.}

\cvitem{[27]}{rs \textbf{O'Donnell, Daniel Paul}, Hobma, Heather*, Cowan, Sandra, Ayers, Gillian*, Bay, Jessica*, Swanepoel, Marinus, Merkley, Wendy, Devine, Kelaine*, Dering, Emma*, and Genee, Inge. 2015. “Aligning Open Access Publication with the Research and Teaching Missions of the Public University.” \textit{Journal of Electronic Publishing} 18.3. Corresponding author. \textsc{doi}: \url{https://doi.org/10.3998/3336451.0018.309}.}

\cvitem{[26]}{r \textbf{O'Donnell, Daniel Paul}. 2013. “‘I certainly have subjects in my mind’: The Diary of Anne Frank as Bildungsroman.” \textit{Canadian Journal of Netherlandic Studies} 32, pp. 49–88. \textsc{doi}: \url{https://doi.org/10.5281/zenodo.3596110}.}

\cvitem{[25]}{r \textbf{O'Donnell, Daniel Paul}. 2012. “Move Over: Learning to Read (and Write) with Novel Technology.” \textit{Scholarly and Research Communication} 3.4. \textsc{doi}: \url{https://doi.org/10.22230/src.2012v3n4a68}.}

\cvitem{[24]}{r \textbf{O'Donnell, Daniel Paul}. 2010. “Different Strokes, Same Folk: Designing the Multi-form Digital Edition.” \textit{Literature Compass} 7.2. \textsc{doi}: \url{https://doi.org/10.1111/j.1741-4113.2009.00683.x}.}

\cvitem{[23]}{r Lee, Stuart and \textbf{O'Donnell, Daniel Paul}. 2009. “From Manuscript to Computer.” In: \textit{Working with Anglo-Saxon Manuscripts}, edited by Owen-Crocker, Gale R. Exeter: Exeter UP, pp. 253–284.}

\cvitem{[22]}{r \textbf{O'Donnell, Daniel Paul}. 2009a. “Back to the future: What digital editors can learn from print editorial practice.” \textit{Literary and Linguistic Computing} 24, pp. 113–125. \textsc{doi}: \url{https://doi.org/10.1093/llc/fqn039}.}

\cvitem{[21]}{i \textbf{O'Donnell, Daniel Paul}. 2009b. “Byte me: Technological Education and the Humanities.” \textit{Heroic Age} 12. \url{http://www.mun.ca/mst/heroicage/issues/12/em.php}.}

\cvitem{[20]}{ri Bodard, Gabriel and \textbf{O'Donnell, Daniel Paul}. 2008. “We are all together: On publishing a Digital Classicist issue of the Digital Medievalist journal.” \textit{Digital Medievalist} 4. Corresponding author. \textsc{doi}: \url{https://doi.org/10.16995/dm.18}.}

\cvitem{[19]}{i \textbf{O'Donnell, Daniel Paul}. 2008. “Resisting The Tyranny of the Screen, or, Must a Digital Edition be Electronic?” \textit{Heroic Age} 11. \url{http://www.heroicage.org/issues/11/em.php}.}

\cvitem{[18]}{r \textbf{O'Donnell, Daniel Paul}. 2007a. “Disciplinary impact and technological obsolescence in digital medieval studies.” In: \textit{A Companion to Digital Literary Studies}, edited by Siemens, Ray and Schreibman, Susan. Cambridge: Blackwell. \textsc{doi}: \url{https://doi.org/10.1002/9781405177504.ch3}.}

\cvitem{[17]}{i \textbf{O'Donnell, Daniel Paul}. 2007b. “If I were ‘You’: How academics can stop worrying and learn to love ‘the encyclopedia that anyone can edit.’” \textit{Heroic Age} 10. \url{http://www.heroicage.org/issues/10/em.html}.}

\cvitem{[16]}{r \textbf{O'Donnell, Daniel Paul}. 2007c. “Material differences: The place of Cædmon’s Hymn in the history of Anglo-Saxon vernacular poetry.” In: \textit{Cædmon’s Hymn and Material Culture in the World of Bede}, edited by Frantzen, Allen J. and Hines, John. Morgantown VA: West Virginia University Press, pp. 15–50.}

\cvitem{[15]}{i \textbf{O'Donnell, Daniel Paul}. 2005a. “O Captain! My Captain! Using technology to guide readers through an electronic edition.” \textit{Heroic Age} 8. \url{http://www.heroicage.org/issues/8/em.html}.}

\cvitem{[14]}{i \textbf{O'Donnell, Daniel Paul}. 2005b. “The Ghost in the Machine: Revisiting an Old Model for the Dynamic Generation of Digital Editions.” \textit{HumanIT} 8.1, pp. 51–71. \textsc{doi}: \url{https://doi.org/10.5281/zenodo.3596125}.}

\cvitem{[13]}{r \textbf{O'Donnell, Daniel Paul}. 2004a. “Bede’s Strategy in Paraphrasing Cædmon’s Hymn.” \textit{JEGP} 103, pp. 417–433.}

\cvitem{[12]}{r \textbf{O'Donnell, Daniel Paul}. 2004b. “Numeric and Geometric Patterning in Cædmon’s Hymn.” \textit{ANQ} 17, pp. 3–12.}

\cvitem{[11]}{i \textbf{O'Donnell, Daniel Paul}. 2004c. “The Digital Medievalist Project.” \textit{Old English Newsletter} 37, pp. 19–21.}

\cvitem{[10]}{i \textbf{O'Donnell, Daniel Paul}. 2004d. “The Doomsday Machine, or, ‘If you build it, will they still come ten years from now?’: What Medievalists working in digital media can do to ensure the longevity of their research.” \textit{Heroic Age} 7. \url{http://www.heroicage.org/issues/7/ecolumn.html}.}

\cvitem{[9]}{r \textbf{O'Donnell, Daniel Paul}. 2003. “‘Pioneers! O Pioneers!’: Some Electronic Editing Do’s and Don’ts from Stijn Streuvels’s De teleurgang van den Waterhoek.” \textit{Literary and Linguistic Computing}.}

\cvitem{[8]}{r \textbf{O'Donnell, Daniel Paul}. 2002a. “Junius’s Knowledge of the Old English Poem Durham.” \textit{Anglo-Saxon England} 30, pp. 231–245.}

\cvitem{[7]}{r \textbf{O'Donnell, Daniel Paul}. 2002b. “The Accuracy of the St. Petersburg Bede.” \textit{Notes and Queries} 247, pp. 4–6.}

\cvitem{[6]}{r \textbf{O'Donnell, Daniel Paul}. 2001. “Fish and Fowl: Generic Expectations and the Relationship between the Old English Phoenix poem and Lactantius’s de ave phoenice.” In: \textit{Germanic Texts and Latin Models: Medieval Reconstructions}, edited by Olsen, K.E., Harbus, A., and Hofstra, T. Germania Latina IV. Mediaevalia Groningana, n.s. 2. Leuven, Paris and Sterling, VA: Peeters, pp. 157–171.}

\cvitem{[5]}{r \textbf{O'Donnell, Daniel Paul}. 1999. “Hædre and Hædre Gehogode (Solomon and Saturn line 62b and Resignation line 63a).” \textit{Notes and Queries} 244, pp. 312–316.}

\cvitem{[4]}{r \textbf{O'Donnell, Daniel Paul}. 1998. “The Spirit and the Letter: Literary Embellishment in Old Frisian Legal Texts.” \textit{Amsterdamer Beiträge zur älteren Germanistik} 49, pp. 245–256.}

\cvitem{[3]}{r \textbf{O'Donnell, Daniel Paul}. 1996. “A Northumbrian Version of ‘Cædmon’s Hymn’ (eordu-recension) in Brussels Bibliothèque Royale Manuscript 8245–57 ff.62r2–v1: Identification, Edition, and Filiation.” In: \textit{Beda Venerabilis: Historian, Monk and Northumbrian}, edited by Houwen, L.A.R.J. and MacDonald, A.A. Groningen: Egbert Forsten, pp. 139–166.}

\cvitem{[2]}{r \textbf{O'Donnell, Daniel Paul}. 1995. “Schoolbook Design in the Fifteenth Century.” \textit{The Yale University Library Gazette} 70, pp. 23–38.}

\cvitem{[1]}{r \textbf{O'Donnell, Daniel Paul}. 1991. “The Collective Sense of Concrete Singular Nouns in Beowulf: Emendations of Sense.” \textit{Neuphilologische Mitteilungen} 92, pp. 433–440.}


\section{Knowledge Mobilisation}

\section{Conference Organisation}

\section {Lectures and Presentations}

\section{Teaching and Supervisory Experience}
\cvitem{\textbf{Summary}}{Daniel Paul O’Donnell has taught undergraduate and graduate courses in English literature, medieval studies, English linguistics, textual scholarship, management information studies, mathematics, computer science, and the digital humanities. His appointments have included Yale, LSU, the University of Lethbridge, and other institutions. He has supervised more than thirty graduate students, including work in scholarly publishing and DH.}
\vspace{1em}
\subsection*{Postdoctoral Supervision}
\cvitem{2021--}{Nathan Woods. Humanities data.}
\vspace{1em}

\subsection*{PhD Supervision}
\cvitem{2025--}{Shara Merrill. Artificial Intelligence and Ethics. Co-supervisor, CPST.}
\cvitem{2024--}{AKM Iftekhar Khalid. Tagorian English vs LLMs. Co-supervisor, CPST.}
\cvitem{2023--}{Frank Onuh. Hate speech and discourse analysis. Co-supervisor, CPST.}
\cvitem{2022--}{Davide Pafumi. Medieval English discourse analysis. Co-supervisor, CPST.}
\vspace{1em}

\subsection*{MA Supervision}
\cvitem{2024--}{Josephine Tabiri. Linguistic Landscapes and Language use in Ghana. Supervisor, English}
\cvitem{2022--}{Morgan Pearce. Chaucer. Co-supervisor, English.}
\cvitem{2021--2023}{Kirandeep Kaur. Language tech. Supervisor, English.}
\cvitem{2021--2023}{Iftekhar Khalid. English in Bangladesh. Supervisor, English.}
\cvitem{2015--2018}{Virgil Grandfield. Creative Non-fiction. Supervisor, Multidisciplinary (withdrawn).}
\cvitem{2014--2017}{Gurpreet Singh. Sikh text editing. Supervisor, Multidisciplinary (withdrawn).}
\cvitem{2013--2016}{Rylan Spenrath. Monstrous fiction. Supervisor, English.}
\cvitem{2011--2014}{Heather Hobma. Ruthwell Kirk. Supervisor, English.}
\cvitem{2005--2008}{Tania Bigthroat. Residential Schools. Co-supervisor, NAS.}
\cvitem{2004--2006}{Shelley Stigter. Native American verbal art. Supervisor, English.}
\vspace{1em}

\subsection*{Committee Memberships (MA/MSc)}
\cvitem{2015--2019}{Mahsa Miri. Global film crediting. Modern Languages/Film.}
\cvitem{2012--2014}{Titi Babalola. Absurd Realism. English.}
\cvitem{2012--2014}{Jessica Bay. Fan Fiction. English.}
\cvitem{2013--}{Fatima Rahman. Region queries. Computer Science.}
\cvitem{2013--2015}{Cody Rioux. Text extraction. Computer Science.}
\cvitem{2009--2011}{Kent Aardse. Digital Literature. English.}
\cvitem{2009--2011}{Drew Luby. Renaissance magic. English.}
\cvitem{2007--2009}{Leanne Little. Shakespeare \& masculinity. English.}
\cvitem{2006--2007}{Rob Meckelborg. Blake and Satan. English.}
\cvitem{2004--2006}{Angela Mlynarski. Text summarization. Computer Science.}
\cvitem{2002--2004}{Laura Cappello. Jane Austen. English.}
\vspace{1em}

\subsection*{University of Lethbridge (1997–)}
\cvitem{Undergraduate}{English 1900, 2100, 2450, 2810, 2900, 3401, 3450, 3601, 3901, 3990, 4400, 4600, 4900}
\cvitem{Graduate}{English 5400 (New Humanities, Digital Humanities, Scholarly Communication), 5600 (Beowulf), 5990 (Independent Studies)}
\cvitem{Other}{Meeting of Minds Journal supervision; Management 4900 (Database Design)}
\vspace{1em}

\subsection*{University of York (1996)}
\cvitem{Undergraduate}{Middle English Paper}
\vspace{1em}

\subsection*{Louisiana State University (1994–1995)}
\cvitem{Undergraduate}{Chaucer, History of English}
\cvitem{Graduate}{Old English, Beowulf}
\vspace{1em}

\subsection*{Yale University (1991–1992)}
\cvitem{TA}{Introduction to the Novel, Age of Johnson, History of English}
\vspace{1em}

\subsection*{Other}
\cvitem{1996}{Workers’ Educational Association: Dark Age Tales}
\cvitem{2009}{ISAS PhD Workshop (co-led with Martin Foys)}
\vspace{1em}


\end{document}

