\documentclass[11pt,letterpaper,sans]{moderncv}

\moderncvstyle{casual}
\moderncvcolor{blue}
\usepackage[scale=0.85]{geometry}

\name{Daniel Paul}{O'Donnell}
\title{Curriculum Vitae}
\email{daniel.odonnell@uleth.ca}
\social[github]{caedmon5}
\social[orcid]{https://orcid.org/0000-0002-0127-4893}
\social[zenodo]{https://zenodo.org/communities/dpodrepository}
\social[home]{https://leth.ca/dspace/handle/10133/3557}

\begin{document}

\makecvtitle

\section{Education}
\cvitem{\textbf{Summary}}{O'Donnell completed his undergraduate studies at the University of Toronto in English and Medieval Latin, followed by an MA and PhD at Yale University in English. His doctoral research focused on variation in Old English poetry, supervised by Fred C. Robinson. \vspace{1em}}

\cventry{1996}{Ph.D. in English}{Yale University}{}{Andrew W. Mellon Fellow; SSHRC Doctoral Fellow; Yale University Fellow \vspace{0.25em} \newline{}\textbf{Dissertation:} \emph{\href{https://doi.org/10.5281/zenodo.1171976}{Manuscript Variation in Multiple Recension Old English Poetic Texts: The Technical Problem and Poetical Art}} \newline{}\textbf{Supervisor:} Fred C. Robinson \newline{}\textbf{Reviewed in} \emph{Linguistica e Filologia} 9 (1999): 156\vspace{0.5em}}{}

\cventry{1991}{M.A. in English}{Yale University}{}{Andrew W. Mellon Fellow; Yale University Fellow\vspace{0.5em}}{}

\cventry{1989}{B.A. in English Language and Literature (with Distinction)}{St. Michael's College, University of Toronto}{}{Various Prizes \vspace{0.25em}\newline{}\textbf{Specialization:} English Language and Literature\newline{}\textbf{Minor:} Medieval Latin (requirements also met for minor in Celtic Studies)}{}
\vspace{1em}

\section{Languages}
\cvitem{\textbf{Summary}}{O'Donnell works in a range of modern and historical languages relevant to medieval studies and digital scholarship. These include English (native), Dutch (spoken and read), French (reading and conversational), and German (reading), as well as a variety of medieval and classical languages, such as Old English, Old Norse, Latin, Old Frisian, and Gothic.}

\vspace{1em}

\cvitem{Modern}{
\begin{itemize}
  \item English (native)
  \item Dutch (preparing for CERF C2 certification)
  \item French (preparing for CERF B2 certification)
  \item German (reading)
\end{itemize}
}

\cvitem{Historical}{Old English, Old Norse, Latin, Old Frisian, Gothic}

\vspace{1em}

\section{Academic Positions}
\cvitem{\textbf{Summary}}{Since 1997, O’Donnell has held a continuing faculty position at the University of Lethbridge, where he currently holds the rank of Professor. He has also taught at Louisiana State University, the University of York, and Yale University, and has held adjunct and affiliate roles at the University of Saskatchewan and within the University
Library. At the University of Toronto, he was the first undergraduate to hold a research position at the Dictionary of Old English.}

\vspace{1em}
\cventry{2019--2024}{Adjunct Professor}{College of Postgraduate and Postdoctoral Studies, University of Saskatchewan}{}{}{}
\cventry{2020--}{Affiliate Member}{Prentice Institute for Global Population, University of Lethbridge}{}{}{}
\cventry{2018}{Professor of English (Offered and Declined)}{University of Saskatchewan}{}{}{}
\cventry{2015--}{Associate Member}{University Library Academic Staff, University of Lethbridge}{}{}{}
\cventry{2010--}{Professor of English (Tenured)}{University of Lethbridge}{}{}{}
\cventry{2002--2010}{Associate Professor of English (Tenured)}{University of Lethbridge}{}{}{}
\cventry{1997--2002}{Assistant Professor of English}{University of Lethbridge}{}{}{}
\cventry{1997}{Tutor}{Department of English and Related Literatures, University of York (UK)}{}{}{}
\cventry{1997}{Tutor}{Workers’ Educational Association (UK)}{}{}{}
\cventry{1994--1995}{Visiting Assistant Professor}{Department of English, Louisiana State University}{}{}{}
\cventry{1991--1992}{Teaching Fellow}{Department of English, Yale University}{}{}{}
\cventry{1987--1989}{Research Assistant}{Dictionary of Old English, University of Toronto}{}{}{}

\vspace{1em}

\section{Academic Leadership}
\cvitem{\textbf{Summary}}{O'Donnell has led multiple academic and professional organisations, including CAFA, ULFA, Force11, GO::DH, and the TEI. He has chaired major digital infrastructure and policy boards, founded scholarly and advocacy groups, and served as department chair at Lethbridge. In these roles, he has consistently taken leadership during moments of structural transition and policy change.}

\vspace{1em}

\cvitem{2024--2025}{\textbf{Past President}, Confederation of Alberta Faculty Associations (CAFA). \vspace{0.15em}\newline{}\textit{Continued strategic advocacy; helped establish quarerly meetings with Deputy Minister and Staff at Ministry of Advanced Education (Alberta); presentation to "Expert Panel on Post-Secondary Institution Funding and Alberta’s Competitiveness (Mintz Panel)}.\vspace{0.25em}}
\cvitem{2023--2024}{\textbf{President}, CAFA. \vspace{0.15em}\newline{}\textit{Assumed role immediately following loss of major member (2/3 of income). Rebuilt organisation and expanded connections to other post-secondary and civil society organisations in province and western Canada as well as nationally. Redesigned public and government relations practices, resulting in improved press-coverage and regular meetings with government officials. Build successful inter-sectoral lobbying effort on Bill 18 (with Rural Municipalities of Alberta and University of Alberta. Helped found Western Regional Council.} \vspace{0.25em}}
\cvitem{2023--2024}{\textbf{Past President}, University of Lethbridge Faculty Association (ULFA). \vspace{0.15em}\newline{}\textit{Supported transition to new leadership; advised on grievance resolution, governance, and bargaining}.}
\cvitem{2021--2023}{\textbf{President}, \textit{ULFA}. Led six-week job action (92\% strike vote, 91\% ratification); rebuilt faculty governance post-strike; expanded communication strategy.}
\cvitem{2020--2021}{\textbf{Vice-President}, \textit{ULFA}. Developed internal governance models; supported bargaining team formation.}
\cvitem{2018--2021}{Chief Spokesperson and Bargaining Chair, ULFA. Negotiated multiple collective agreements (including Navitas MOU); launched and secured approval for job action fund.}
\cvitem{2018--2019}{President, Force11. Oversaw launch of FORCE11 Scholarly Communication Institute (FSCI); expanded global FAIR and Open Science policy advocacy.}
\cvitem{2013--2017}{Vice-President, Force11. Built global partnerships in scholarly communication and infrastructure.}
\cvitem{2015--2017}{Founding Director, Lethbridge Centre for the Study of Scholarly Communication. Created institutional centre supporting interdisciplinary research and publication innovation.}
\cvitem{2012--2015}{Founding Chair, Global Outlook::Digital Humanities (GO::DH). Founded equity-centred digital humanities network; initiated multilingual and regional DH programs.}
\cvitem{2010--2013}{Co-President, Canadian Society for Digital Humanities (CSDH/SCHN). Coordinated national DH strategy and membership growth.}
\cvitem{2009--2012}{Chair, Digital Initiatives Advisory Board, Medieval Academy of America. Guided MAA infrastructure strategy; advised on journal and archive digitisation.}
\cvitem{2006--2011}{Chair and CEO, Text Encoding Initiative (TEI). Restructured TEI governance; launched member-led consortium; resolved internal staff conflict; published revised TEI Guidelines.}
\cvitem{2003--2009}{Founding Director, Digital Medievalist. Established peer-reviewed journal and infrastructure for medievalists working in digital media.}
\cvitem{2005--2008, 2023--}{Chair, Department of English, University of Lethbridge. Led department through curriculum overhaul, review, and hiring plan during institutional crisis; implemented new governance model.}

\end{document}
